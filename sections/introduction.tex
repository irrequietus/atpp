%%% 
%%% Annotated C++ template parameter packs
%%% 
%%% Copyright (C) 2014, George Makrydakis irrequietus@gmail.com>
%%%
%%% This document is released under a Creative Commons 4.0 license, specifically
%%% Attribution-NonCommercial-NoDerivatives 4.0 International. Details for the
%%% aforementioned license at http://creativecommons.org/licenses/by-nc-nd/4.0/
%%% This repository's "license.txt" contains the full text of said license.
%%%
%%% This document draft discusses the possibility of extending C++ template
%%% parameter packs with optional syntax described as "annotation". The only
%%% git repository for tracking real-time changes by the original author is
%%% at https://github.com/irrequietus/atpp.
%%%
%%% Periodically, the author ships major/important revisions of the draft in
%%% PDF format, using the project website at http://atpp.irrequietus.eu with
%%% each PDF file having a specific sha1sum checksum.
%%%
%%% This section is part of said document.

\subsection{How to read this document}

\p If you are interested in quick examples of the syntax itself, focus on introduction, parameter pack annotation semantics and deployment scenarios.
The sections alluding to "formal" proofs of the equivalent forms will be completed and moved to an addendum to the paper in its final N-numbered form.
Ternary annotation refers to the generative aspects of annotation itself and is being currently prepared.
The interplay between concepts \cite{Stroustrup2012,Sutton2013} and how pack annotation works \textit{with} them, enhancing \textit{them} without \textit{them} being able to supplant annotation (and any other kind of pack processing proposal related to EWG issue 30 \cite{Abrahams2012} like this one) is being formally processed (boring to read) and will be merged later on.
The goal is to produce a very compact proposal document with an addendum related to details beyond code sample exposition.
Paragraphs on addressing several concerns are still being added so watch the official git repository of the document and its issue tracker.

\subsection{Making parameter packs first class citizens in a template parameter list}

\p Template parameter packs are product types, n-tuples of parameters of the same parameter type.
In all levels of C++ template meta-programming, recursive template instantiations can be shown to play the role of algebraic values, used in cohort with class template partial specializations, function template overloads, SFINAE, ADL, partial ordering, as template parameter list processing and type-safe code generation tools \cite{Munoz2008,JarviWL03,Abrahams2004,Alexandrescu2001}.
Such techniques are critical to fully exploiting C++ template meta-programming \textit{pattern matching} characteristics typically belonging to functional programming languages.

\p While parameter packs greatly enhance the functional nature of C++ template meta-programming \cite{Alexandrescu2012}, the lack of language level pack processing sfeatures related to both access and generation has been partially been addressed through template meta-programming library facilities out of which variadics had originally emerged as a proposal \cite{Czarnecki2000,Alexandrescu2001,Abrahams2004,Gregor2006,Gregor2008,Gregor2008a}.
When more complex constructs depending on parameter pack processing features became the norm in C++ libraries of widespread use, the limitations of the library approach became quite evident.

\p Of note, recourse to classical techniques in such libraries, as the aforementioned recursive template instantiations, proved onerous enough to stem C++ EWG issues and proposals at a language feature level \cite{Abrahams2012,Middleditch2013,Wakely2013,Wakely2013a} that aim to \textit{greatly enhance template meta-programming itself}; such bibliography offers ample motivation and examples of interest to the reader, making library-level reccomendations for such features confusing and fears of increasing complexity of template meta-programming objectively unfounded.
Importantly, they may even ease the implementation of purely functional programming constructs for template meta-programming like \textit{monads} \cite{Porkolab2010,Sinkovich2013} for purposes including better error reporting \cite{Sinkovich2013}.

\p The \textit{actual} effect of adopting pack processing features at a language level is to turn constructs like \textit{typelists} \cite{Czarnecki2000,Alexandrescu2001,Abrahams2004} into first class citizens within template parameter lists.
In combination with \textit{concepts} \cite{Stroustrup2012,Sutton2013} (constraints that do not overlap with the actual intention of annotation), such features can almost totally eliminate the need for laborious source code boilerplate in vertical and horizontal dimensions, where even preprocessor meta-programming seems at times inevitable.

\section{Getting quickly acquainted with the syntax}
\p \textit{Annotators} are tied up to the \textit{triple-dot} operator and pack identifier as a logical extension of the pack concept; there is where dealing with C++ abstract syntax tree building has finished addressing code strictly specific to a parameter pack.
The curly braces (i.e. \textit{interval} annotator) have been selected because \textit{they are not used currently in the context of a template parameter list} following pack identifiers or \textit{triple-dot} operators and are therefore guaranteed to provide unambiguous parsing.
These can be followed by an additional annotator in square brackets (i.e. \textit{pattern annotator}) that can only be present if the \textit{interval} annotator follows a pack identifier or a \textit{triple-dot} operator.
Given that all standard C++ compilers issue errors when parameter packs are not expanded through the \textit{triple-dot} operator, the presence of a single integral constant expression within the \textit{interval} annotator (i.e. $\{\}$) is used as index based access for the elements of a parameter pack, \textit{omitting the triple-dot} and thus providing an unambiguous parse for the pack at minimal cost for an implementor.

\p A series of initial examples are offered in order to familiarize the reader with the syntax; its coherence and unambiguity are analyzed later on.
The full syntax is used in each occasion and the reader should be aware that these represent \text{optional syntactic extensions} to pack notation but in order for annotation to be valid, the annotators have to be present in a certain order.
The reason for this is that there are equivalent forms with non-annotated (C++11/14) parameter packs.
Any combination of integral constant expressions that is considered invalid within annotation context, removes the related template from the matching candidates.

\begin{minted}{c++}
// N,M,K are integral constant expressions
template<typename... T{N,M}[K]> struct class_template
{ /*... */ };

// X is an integral constant expression such as N <= X < M
template<typename... T> struct class_template<T...{X}>
{ typedef T{Z} type; /*... */ };

// [I,J) is any valid subinterval of [N,M), I <= L < J
template<typename... T> struct class_template<T...{I,J}>
{ typedef T{L} type; /*... */ };

// A =< K < B
template<typename... P{A,B}[C]> void function_template(P...)
{ typedef P{K} type; /*... */ };

// S <= U < D
template<typename... W{S,D}[F]> void function_template(W...)
{ W{U} varname; /*... */ };

\end{minted}

\p Future revisions of this draft will display similar effects of the same \textit{pack annotation} semantics for non-type and template-type template parameters and their interaction with concepts \cite{Stroustrup2012,Sutton2013}.
\newpage

\subsection{The rules and syntax of annotation}

\p In the following syntax lookup tables \textcolor{Magenta}{magenta} colored elements are positive integral constant expressions.
The letter $T$ is used as a pack identifier alluding to an n-ordered sequence of parameter types of the same type (non-type template parameters, type-type template parameters, template-type template parameters) whose size and and eventual pattern consistency are specified by annotation semantics.
Given that the pack identifier is specified prior to annotation, it may be used in constant expressions from which the annotation integral constant expressions are derived themselves; this makes \textit{equivalent forms} with non-annotated parameter packs possible thus making annotated template parameter packs a higher level abstraction of the pack concept itself.
This is analyzed later on, for now, let's give a look at the syntax table.

\begin{tabularx}{\textwidth}{l|c|X}
  \textbf{Syntax} & \textbf{Context}  &\textbf{Significance} \\
\hline
$\bm{...T\{\textcolor{Magenta}{\textcolor{Magenta}{N}},\textcolor{Magenta}{M}\}[\textcolor{Magenta}{K}]}$ & declaration & $\bm{sizeof...(T)} \in [\textcolor{Magenta}{N},\textcolor{Magenta}{M})$, \textit{any}-tuple of first $sizeof...(T)/K$ parameters \\
$\bm{...T\{\textcolor{Magenta}{N}\}[\textcolor{Magenta}{K}]}$ & declaration & $\bm{sizeof...(T)} == \textcolor{Magenta}{N}$, \textit{any}-tuple of first $sizeof...(T)/K$ parameters \\
$\bm{T\{\textcolor{Magenta}{N}\}}$ & expansion & access parameter at index $\textcolor{Magenta}{N}$ in pack $T$ \\
$\bm{T\{\textcolor{Magenta}{N},\textcolor{Magenta}{M}\}[\textcolor{Magenta}{K}]}$ & expansion & $T\{\textcolor{Magenta}{N}\},T\{\textcolor{Magenta}{N+1}\},...,T\{\textcolor{Magenta}{M-1}\}$ expanded $\textcolor{Magenta}{K}$ times \\
$\bm{T...\{\textcolor{Magenta}{N},\textcolor{Magenta}{M}\}[\textcolor{Magenta}{K}]}$ & expansion & $\bm{sizeof...(T)} \in [\textcolor{Magenta}{N},\textcolor{Magenta}{M})$, then expand that $T$ for \textcolor{Magenta}{K} times   \\
$\bm{T...\{\textcolor{Magenta}{N}\}[\textcolor{Magenta}{K}]}$ & expansion & $T\{\textcolor{Magenta}{0}\},T\{\textcolor{Magenta}{1}\},T\{\textcolor{Magenta}{2}\},...,T\{\textcolor{Magenta}{N-1}\}$, then expand that for $\textcolor{Magenta}{K}$ times\\
\end{tabularx}

\p The constraints over the values of integral constant expressions become obvious when we start approaching parameter pack access in order to resolve EWG\#30 \cite{Abrahams2012}, while the generative features annotated packs possess make them distinct in their primary intention of use to \textit{concepts}; the latter are actually \textit{enhanced} by annotated packs.
We have three \textit{annotators}:
\begin{enumerate}
\item\p The \textcolor{Magenta}{\textbf{\textit{size}}} annotator provides a match for any pack whose \textit{size} is within the left-closed, right-open interval specified by two comma separated, curly-brace enclosed integral constant expressions serving as endpoints, following the \textit{triple-dot} pack specifier.
When only one expression is enclosed, a match is provided only for a pack whose $\bm{sizeof...(T)}$ equals the expression.
When no expression is explicitly specified, $\bm{sizeof...(T)}$ is the implied \textit{equivalent} and the \textit{size} annotator may be omitted if not followed by the \textit{tuple} annotator.
In \textit{declarative context}, the pack identifier is specified within \textit{triple-dot} and the annotator; in \textit{expansive context}, the pack identifier precedes the \textit{triple-dot} in full compliance with current parameter pack semantics.
Whenever used in \textit{expansive context}, any expressions related to \textit{size} annotation must express pack expansion to a size within the interval specified in declaration.

\item\p The \textcolor{Magenta}{\textbf{\textit{range}}} annotator is strictly used in \textit{expansive context}, as either one or two comma separated and curly brace enclosed integral constant expressions immediately following an already specified pack identifier, without an accompanying \textit{triple-dot} specifier.
With one expression, it expands to a single parameter contained in the original pack whose \textit{index} is the enclosed expression (i.e. $\bm{T\{N\}}$).
When two expressions are used, they serve as endpoints of a left-closed, right-open interval of \textit{indices} to which the pack expands (i.e. $\bm{T\{N,M\} \equiv T\{N\},T\{N+1\},T\{N+2\},...T\{M-1\}}$).

\item\p The \textcolor{Magenta}{\textbf{\textit{tuple}}} annotator is an omissible (\textit{strictly following curly-braced annotation}) square bracket enclosed integral constant expression whose implied value is $\bm{1}$.
In \textit{declarative context} it has reflective properties upon the types comprising the pack, since it requires the pack to be an exact multiple of its first $\bm{sizeof...(T)/k}$ parameters  with \textbf{\textit{k}} being the enclosed integral constant expression.
In \textit{expansive context}, it repeats the annotated pack as many times as said expression.

\end{enumerate}
\p Annotation by \textit{size} with deducible endpoint expressions can allow specification of template parameter lists where \textit{multiple} size-annotated parameter packs coexist in declarative context with even non-annotated packs.
\textcolor{RoyalBlue}{When multiple size-annotated packs are present, futher work is needed to specify when deducibility is ensured (on the to-do list of the current branch) otherwise ambiguity ensues.}

\begin{minted}{c++}
/* 1: there is a way to allow this but complexity must be justified and properly
 *    worded to make a point for or against it */
template<typename... X{5,10}, typename... Y{3,5}> struct sample1 {}; /* see 1 */
/* 2: these are more than ok. */
template<typename... X{5,10}, typename... Y>      struct sample2 {}; /* ok */
template<typename... X{10}, typename... Y>        struct sample3 {}; /* ok */
\end{minted}


\p Annotators thus allow index and interval-based parameter pack semantics to be used in both declaration (for matching purposes) and expansion (for matching and generative purposes) and address directly \textbf{EWG\#30} \cite{Abrahams2012} using terse, unambiguous and coherent syntax that graciously degenerates into parameter packs as we know them today.
This is by design so that they can be easily submitted to \textit{partial ordering}, a well known matching criterion in C++ between them and regular parameter packs.
Any annotation where an invalid interval is specified, removes the template containing such pack from the matching candidates status, which is the SFINAE \cite{sfinae} effect in this case.

\subsection{Ternary Annotation and Generative Constructs}
    %%% 
%%% Annotated C++ template parameter packs
%%% 
%%% Copyright (C) 2014, George Makrydakis irrequietus@gmail.com>
%%%
%%% This document is released under a Creative Commons 4.0 license, specifically
%%% Attribution-NonCommercial-NoDerivatives 4.0 International. Details for the
%%% aforementioned license at http://creativecommons.org/licenses/by-nc-nd/4.0/
%%% This repository's "license.txt" contains the full text of said license.
%%%
%%% This document draft discusses the possibility of extending C++ template
%%% parameter packs with optional syntax described as "annotation". The only
%%% git repository for tracking real-time changes by the original author is
%%% at https://github.com/irrequietus/atpp.
%%%
%%% Periodically, the author ships major/important revisions of the draft in
%%% PDF format, using the project website at http://atpp.irrequietus.eu with
%%% each PDF file having a specific sha1sum checksum.
%%%
%%% This section is part of said document.

{\color{magenta}{\textit{Currently developing exposition.
Formal thinking comes first, but a series of quick examples will be properly adjusted for immediate presentation in earlier sections.
This should get a more descriptive name.}}}

        

