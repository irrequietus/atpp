
\subsection{The rules and syntax of annotation}

\p In the following syntax lookup tables \textcolor{Magenta}{magenta} colored elements are positive integral constant expressions.
The letter $T$ is used as a pack identifier with $|\overline{T}|$ equivalent to $sizeof...(T)$, alluding to a sequence of parameter types of the same type (non-type template parameters, type-type template parameters, template-type template parameters).
Given that the pack identifier is specified prior to annotation, it may be used in constant expressions from which the annotation integral constant expressions are derived themselves; this makes \textit{equivalent forms} with non-annotated parameter packs possible thus making annotated template parameter packs a higher level abstraction of the pack concept itself.
This is analyzed later on, for now, let's give a look at the syntax table.

\begin{tabularx}{\textwidth}{l|c|X}
  \textbf{Annotation Syntax} & \textbf{Context}  &\textbf{Significance} \\
\hline
$\bm{...T\{\textcolor{Magenta}{\textcolor{Magenta}{N}},\textcolor{Magenta}{M}\}[\textcolor{Magenta}{K}]}$ & declaration & $\bm{sizeof...(T)} \in [\textcolor{Magenta}{N},\textcolor{Magenta}{M})$, is K-tuple of the same parameter sequence \\
$\bm{...T\{\textcolor{Magenta}{N}\}[\textcolor{Magenta}{K}]}$ & declaration & $\bm{sizeof...(T)} == \textcolor{Magenta}{N}$ , is K-tuple of the same parameter sequence \\
$\bm{T\{\textcolor{Magenta}{N}\}}$ & expansion & access type by index $\textcolor{Magenta}{N}$ in $T$ \\
$\bm{T\{\textcolor{Magenta}{N}\}...}$ & expansion& $T\{\textcolor{Magenta}{N}\},T\{\textcolor{Magenta}{N+1}\},T\{\textcolor{Magenta}{N+2}\},\dots{},T\{\textcolor{Magenta}{sizeof...(T)-1}\}$ \\
$\bm{T\{\textcolor{Magenta}{N},\textcolor{Magenta}{M}\}[\textcolor{Magenta}{K}]}$ & expansion & $T\{\textcolor{Magenta}{N}\},T\{\textcolor{Magenta}{N+1}\},...,T\{\textcolor{Magenta}{M-1}\}$ repeated $\textcolor{Magenta}{K}$ times \\
$\bm{T...\{\textcolor{Magenta}{N},\textcolor{Magenta}{M}\}[\textcolor{Magenta}{K}]}$ & expansion & $\bm{sizeof...(T)} \in [\textcolor{Magenta}{N},\textcolor{Magenta}{M})$ repeated $\textcolor{Magenta}{K}$ times \\
$\bm{T...\{\textcolor{Magenta}{N},\textcolor{Magenta}{M}\}}$ & expansion & $\bm{sizeof...(T)} \in [\textcolor{Magenta}{N},\textcolor{Magenta}{M})$ repeated $\textcolor{Magenta}{K}$ times \\
$\bm{T...\{\textcolor{Magenta}{N}\}[\textcolor{Magenta}{K}]}$ & expansion & $\bm{sizeof...(T)}\geq{\textcolor{Magenta}{N}}$, to $T\{\textcolor{Magenta}{0}\},T\{\textcolor{Magenta}{1}\},T\{\textcolor{Magenta}{2}\},...,T\{\textcolor{Magenta}{N-1}\}$ $\textcolor{Magenta}{K}$ times\\
\end{tabularx}

\p The constraints over the values of integral constant expressions become obvious when we start approaching parameter pack access in order to resolve EWG\#30 \cite{Abrahams2012}, while the generative features annotated packs possess make them distinct in their primary intention of use to \textit{concepts}; the latter are actually \textit{enhanced} by annotated packs.
We thus have three kinds of annotation, as in the following rules:
\begin{enumerate}
\item\p The \textcolor{Magenta}{\textbf{\textit{interval}}} annotator contains none, one or two comma separated positive integral constant expressions.
When no expression is specified, $sizeof...(T)$ is implied as an equivalent.
If and only if it is not followed by the \textit{pattern} annotator and $sizeof...(T)$ is implied, it may be omitted since it represents a form equivalent to a non-annotated template parameter pack.
In declaration with a single positive integral constant expression, a match is provided for a pack whose $sizeof...(T)$ is identical to said expression; with two, a match is provided for any compatible parameter sequence whose size is within the specified interval.
In expansion with a single positive integral constant expression, it expands the parameter pack it accompanies to a size equal to said expresion beginning from the leftmost parameter contained;
when used with two, it provides a valid match for any compatible parameter sequence of size within the specified interval.

\item\p The \textcolor{Magenta}{\textbf{\textit{pattern}}} annotator can contain none or one positive integral constant expressions.
When no expression is specified, $\bm{1}$ is implied and it may be omitted since it represents a form equivalent to an \textit{interval}-annotated pack.
In declaration, the pattern annotator requires the pack it is applied upon to be an exact multiple of an ordered sequence of types, the latter of size equal to the integral constant expression it encloses. In expansion, it repeats the annotated pack as many times as said expression.
The \textit{pattern} annotator is a valid construct if and only if preceded by a valid \textit{interval} annotator.

\item\p The \textcolor{Magenta}{\textbf{\textit{access}}} annotator is strictly for expansion; it is comprised of one or two comma separated positive integral constant expressions enclosed in curly braces immediately following a pack identifier (i.e. $T\{N\}$ and $T\{N,M\}$).
With one expression, it expands to a single parameter contained in the original pack at a position specified by the enclosed expression (i.e. $T\{N\}$).
With two expressions, it expands to the ordered sequence of parameters between the two integral constant expressions, meaning left-closed right-open interval (i.e. $\bm{T\{N,M\} \equiv T\{N\},T\{N+1\},T\{N+2\},...T\{M-1\}}$).
When a single expression access annotator is followed by a \textit{triple-dot} the effect is equivalent to two-expression access annotation where the second expression is $sizeof...(T)$ (i.e. $\bm{T\{N\}... \equiv T\{N,sizeof...(T)\}}$).
\end{enumerate}

\p Annotators thus allow index and interval-based parameter pack semantics to be used in both declaration (for matching purposes) and expansion (for matching and generative purposes) and address directly \textbf{EWG\#30} \cite{Abrahams2012} using terse, unambiguous and coherent syntax that graciously degenerates into parameter packs as we know them today.
This is by design so that they can be easily submitted to \textit{partial ordering}, a well known matching criterion in C++ between them and regular parameter packs.
Any annotation where an invalid interval is specified, removes the template containing such pack from the matching candidates status, which is the SFINAE \cite{sfinae} effect in this case.

\subsection{Ternary Annotation and Generative Constructs}
    %%% 
%%% Annotated C++ template parameter packs
%%% 
%%% Copyright (C) 2014, George Makrydakis irrequietus@gmail.com>
%%%
%%% This document is released under a Creative Commons 4.0 license, specifically
%%% Attribution-NonCommercial-NoDerivatives 4.0 International. Details for the
%%% aforementioned license at http://creativecommons.org/licenses/by-nc-nd/4.0/
%%% This repository's "license.txt" contains the full text of said license.
%%%
%%% This document draft discusses the possibility of extending C++ template
%%% parameter packs with optional syntax described as "annotation". The only
%%% git repository for tracking real-time changes by the original author is
%%% at https://github.com/irrequietus/atpp.
%%%
%%% Periodically, the author ships major/important revisions of the draft in
%%% PDF format, using the project website at http://atpp.irrequietus.eu with
%%% each PDF file having a specific sha1sum checksum.
%%%
%%% This section is part of said document.

{\color{magenta}{\textit{Currently developing exposition.
Formal thinking comes first, but a series of quick examples will be properly adjusted for immediate presentation in earlier sections.
This should get a more descriptive name.}}}

        
