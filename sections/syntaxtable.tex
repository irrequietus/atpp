
\subsection{The rules and syntax of annotation}

\p Annotators are balanced enclosures ($\bm{\{\}}$ and $\bm{\{\}[]}$) containing integral constant expressions in an ordered syntactic relation with the \textit{triple-dot} and the identifier used for a parameter pack.
Annotation is the use of annotators in a specific, declarative or expansive, parameter pack context.
The result of annotation is to allow parameter access by a specific index or over a range of indices in a pack (access annotation), specify size constraints of a specific size or over a range of sizes for a pack (size annotation), provide repetitive pattern constraints upon declaration or generating patterns through repetition during expansion (tuple annotation).
Any annotation where an invalid interval is specified, removes the template containing such pack from the matching candidates status, which is the SFINAE \cite{sfinae} effect in this case.

\subsubsection{Annotation syntax table}

\p In the following syntax lookup tables \textcolor{Magenta}{magenta} colored elements are positive integral constant expressions.
The letter $T$ is a C++ parameter pack (non-type template parameters, type-type template parameters, template-type template parameters) submitting to the constraints of annotation.
Given that the pack identifier is specified prior to annotation, it may be used in constant expressions from which the annotation integral constant expressions are derived themselves; this makes \textit{equivalent forms} with non-annotated parameter packs possible thus making annotated template parameter packs a higher level abstraction of the pack concept itself.

\begin{tabularx}{\textwidth}{l|c|X}
  \textbf{Annotation} & \textbf{Context}  &\textbf{Significance} \\
\hline
$\bm{...T\{\textcolor{Magenta}{\textcolor{Magenta}{N}},\textcolor{Magenta}{M}\}[\textcolor{Magenta}{K}]}$ & declaration & $\bm{sizeof...(T)} \in [\textcolor{Magenta}{N},\textcolor{Magenta}{M})$, \textit{any}-tuple of first $sizeof...(T)/K$ parameters \\
$\bm{...T\{\textcolor{Magenta}{N}\}[\textcolor{Magenta}{K}]}$ & declaration & $\bm{sizeof...(T)} == \textcolor{Magenta}{N}$, \textit{any}-tuple of first $sizeof...(T)/K$ parameters \\
$\bm{T...\{\textcolor{Magenta}{N},\textcolor{Magenta}{M}\}[\textcolor{Magenta}{K}]}$ & expansion & $\bm{sizeof...(T)} \in [\textcolor{Magenta}{N},\textcolor{Magenta}{M})$, then expand that ${\bm{T}}$ for \textcolor{Magenta}{K} times   \\
$\bm{T...\{\textcolor{Magenta}{N}\}[\textcolor{Magenta}{K}]}$ & expansion & $\bm{sizeof...(T) == N}$, then expand that ${\bm{T}}$ for $\textcolor{Magenta}{K}$ times\\
$\bm{T\{\textcolor{Magenta}{N}\}}$ & expansion & access parameter at index $\textcolor{Magenta}{N}$ in pack $T$ \\
$\bm{T\{\textcolor{Magenta}{N}\}[\textcolor{Magenta}{K}]...}$ & expansion & access parameter at index $\textcolor{Magenta}{N}$ in pack $T$, expand \textcolor{Magenta}{K} times\\
$\bm{T\{\textcolor{Magenta}{N},\textcolor{Magenta}{M}\}[\textcolor{Magenta}{K}]...}$ & expansion & access $T\{\textcolor{Magenta}{N}\},T\{\textcolor{Magenta}{N+1}\},...,T\{\textcolor{Magenta}{M-1}\}$ expanded $\textcolor{Magenta}{K}$ times \\
\end{tabularx}

\p The constraints over the values of integral constant expressions become obvious when we start approaching parameter pack access in order to resolve EWG\#30 \cite{Abrahams2012}, while the generative features annotated packs possess make them distinct in their primary intention of use to \textit{concepts}; the latter are actually \textit{enhanced} by annotated packs.
We have three kinds of \textit{annotation}:
\begin{enumerate}
\item\p \textcolor{Magenta}{\textbf{\textit{size}}} annotation provides a match for any pack whose \textit{size} is within the left-closed, right-open interval specified by two comma separated, curly-brace enclosed integral constant expressions serving as endpoints, following the \textit{triple-dot} pack specifier.
When only one expression is enclosed, a match is provided only for a pack whose $\bm{sizeof...(T)}$ equals the expression.
When no expression is explicitly specified, $\bm{sizeof...(T)}$ is the implied \textit{equivalent} and the \textit{size} annotator may be omitted if not followed by the \textit{tuple} annotator.
In \textit{declarative context}, the pack identifier is specified within \textit{triple-dot} and the annotator; in \textit{expansive context}, the pack identifier precedes the \textit{triple-dot} in full compliance with current parameter pack semantics.
Whenever used in \textit{expansive context}, any expressions related to \textit{size} annotation must express pack expansion to a size within the interval specified in declaration.

\item\p \textcolor{Magenta}{\textbf{\textit{access}}} annotation is strictly used in \textit{expansive context}, as either one or two comma separated and curly brace enclosed integral constant expressions immediately following an already specified pack identifier.
With one expression, it expands to a single parameter contained in the original pack whose \textit{index} is the enclosed expression (i.e. $\bm{T\{N\}}$). If it is followed by a \textit{tuple} annotator, it must be succeeded by a \textit{triple-dot} and is equivalent to expanding that particular parameter as many times as the integral constant expression in the \textit{tuple} annotator.
When two expressions are used, they serve as endpoints of a left-closed, right-open interval of \textit{indices} from which parameters from the pack are to be expanded and it must always be followed by a \textit{triple-dot} (i.e. $\bm{T\{N,M\}[K]... \equiv T\{N\},T\{N+1\},T\{N+2\},...,T\{M-1\}} \times K $).

\item\p \textcolor{Magenta}{\textbf{\textit{tuple}}} annotation is an omissible (\textit{strictly following curly-braced annotators}) square bracket enclosed integral constant expression whose implied value is $\bm{1}$.
In \textit{declarative context} it has reflective properties upon the types comprising the pack, since it requires the pack to be an exact multiple of its first $\bm{sizeof...(T)/k}$ parameters  with \textbf{\textit{k}} being the enclosed integral constant expression.
In \textit{expansive context}, it repeats the annotated pack as many times as said expression.
\end{enumerate}

\subsubsection{Equivalent forms and other trivia}
\p Given annotator definition and the same conventions followed in the annotation syntax table, the following equivalent forms hold true:

\begin{tabularx}{\textwidth}{l | l l l}
  \textbf{Context} & \textbf{Forms} \\
\hline
   ...T & ...T\{\}    & ...T\{sizeof...(T)\}     & ...T\{0,sizeof...(T)\} \\
        & ...T\{\}[]  & ...T\{sizeof...(T)\}[]   & ...T\{0,sizeof...(T)\}[] \\
        & ...T\{\}[1] & ...T\{sizeof...(T)\}[1] & ...T\{0,sizeof...(T)\}[1] \\

   T... & T...\{\} & T...\{sizeof...(T)\} & T...\{0,sizeof...(T)\} \\
        & T...\{\}[] & T...\{sizeof...(T)\}[] & T...\{0,sizeof...(T)\}[] \\
        & T...\{\}[1] & T...\{sizeof...(T)\}[1] & T...\{0,sizeof...(T)\}[1] \\
        & T\{0,sizeof...(T)\}... & T\{0,sizeof...(T)\}[]... & T\{0,sizeof...(T)\}[1]... \\
\hline
 ...T\{N,M\} & ...T\{N,M\}[] & ...T\{N,M\}[1] \\
\hline
   T\{N,M\}... & T\{N,M\}[]... & T\{N,M\}[1]... \\
\hline
   T\{N\}[]... & T\{N\}[1]... \\
\end{tabularx}

\p The interesting thing from this table is that there is a semantic overlap between ranged size and ranged access annotation in expansion when the endpoints specify a [0,sizeof...(T)) interval;
in the first case you are asking for a match for a sizeof...(T) sized pack, in the second case you are generating the same pack because [0,sizeof...(T)) is identical to T.

\p Note that there is no T\{N\}... equivalent because that is used for declaring a non-type parameter pack whose non-type parameters are of type T\{N\}, given an already declared pack T.
\textcolor{RoyalBlue}{This should be a nice place to put other clarifications in respect to
array type definitions based on T\{N\}.
Consider adding the \textit{applicative} features of the C++11/14 \textit{triple-dot} in this section as an example as well.
Could at this point get rid of the "math stuff everybody dislikes reading"...}

\subsubsection{Multiple annotated parameter pack convention}
\p Annotation by \textit{size} with deducible endpoint expressions can allow specification of template parameter lists where \textit{multiple} size-annotated parameter packs coexist in declarative context with even non-annotated packs.
\textcolor{RoyalBlue}{When multiple size-annotated packs are present, futher work is needed to specify when deducibility is ensured (on the to-do list of the current branch) otherwise ambiguity ensues.}

\begin{minted}{c++}
/* 1: there is a way to allow this but complexity must be justified and properly
 *    worded to make a point for or against it */
template<typename... X{5,10}, typename... Y{3,5}> struct sample1 {}; /* see 1 */
/* 2: these are more than ok. */
template<typename... X{5,10}, typename... Y>      struct sample2 {}; /* ok */
template<typename... X{10}, typename... Y>        struct sample3 {}; /* ok */
\end{minted}


\p Annotators thus allow index and interval-based parameter pack semantics to be used in both declaration (for matching purposes) and expansion (for matching and generative purposes) and address directly \textbf{EWG\#30} \cite{Abrahams2012} using terse, unambiguous and coherent syntax that graciously degenerates into parameter packs as we know them today.
This is by design so that they can be easily submitted to \textit{partial ordering}, a well known matching criterion in C++ between them and regular parameter packs.
        
