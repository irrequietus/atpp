
\subsection{The rules and syntax of annotation}

\p In the following syntax lookup tables \textcolor{Magenta}{magenta} colored elements are positive integral constant expressions.
The letter $T$ is used as a pack identifier alluding to an n-ordered sequence of parameter types of the same type (non-type template parameters, type-type template parameters, template-type template parameters) whose size and and eventual pattern consistency are specified by annotation semantics.
Given that the pack identifier is specified prior to annotation, it may be used in constant expressions from which the annotation integral constant expressions are derived themselves; this makes \textit{equivalent forms} with non-annotated parameter packs possible thus making annotated template parameter packs a higher level abstraction of the pack concept itself.
This is analyzed later on, for now, let's give a look at the syntax table.

\begin{tabularx}{\textwidth}{l|c|X}
  \textbf{Annotation Syntax} & \textbf{Context}  &\textbf{Significance} \\
\hline
$\bm{...T\{\textcolor{Magenta}{\textcolor{Magenta}{N}},\textcolor{Magenta}{M}\}[\textcolor{Magenta}{K}]}$ & declaration & $\bm{sizeof...(T)} \in [\textcolor{Magenta}{N},\textcolor{Magenta}{M})$, \textcolor{Magenta}{K}-tuple of first $sizeof...(T)/K$ parameters \\
$\bm{...T\{\textcolor{Magenta}{N}\}[\textcolor{Magenta}{K}]}$ & declaration & $\bm{sizeof...(T)} == \textcolor{Magenta}{N}$, \textcolor{Magenta}{K}-tuple of first $sizeof...(T)/K$ parameters \\
$\bm{T\{\textcolor{Magenta}{N}\}}$ & expansion & access parameter at index $\textcolor{Magenta}{N}$ in pack $T$ \\
$\bm{T\{\textcolor{Magenta}{N},\textcolor{Magenta}{M}\}[\textcolor{Magenta}{K}]}$ & expansion & $T\{\textcolor{Magenta}{N}\},T\{\textcolor{Magenta}{N+1}\},...,T\{\textcolor{Magenta}{M-1}\}$ expanded $\textcolor{Magenta}{K}$ times \\
$\bm{T...\{\textcolor{Magenta}{N},\textcolor{Magenta}{M}\}[\textcolor{Magenta}{K}]}$ & expansion & $\bm{sizeof...(T)} \in [\textcolor{Magenta}{N},\textcolor{Magenta}{M})$, then expand that $T$ for \textcolor{Magenta}{K} times   \\
$\bm{T...\{\textcolor{Magenta}{N}\}[\textcolor{Magenta}{K}]}$ & expansion & $T\{\textcolor{Magenta}{0}\},T\{\textcolor{Magenta}{1}\},T\{\textcolor{Magenta}{2}\},...,T\{\textcolor{Magenta}{N-1}\}$, then expand that for $\textcolor{Magenta}{K}$ times\\
\end{tabularx}

\p The constraints over the values of integral constant expressions become obvious when we start approaching parameter pack access in order to resolve EWG\#30 \cite{Abrahams2012}, while the generative features annotated packs possess make them distinct in their primary intention of use to \textit{concepts}; the latter are actually \textit{enhanced} by annotated packs.
We have three \textit{annotators}:
\begin{enumerate}
\item\p The \textcolor{Magenta}{\textbf{\textit{size}}} annotator provides a match for any pack whose \textit{size} is within the left-closed, right-open interval specified by two comma separated, curly-brace enclosed integral constant expressions serving as endpoints, following the \textit{triple-dot} pack specifier.
When only one expression is enclosed, a match is provided only for a pack whose $\bm{sizeof...(T)}$ equals the expression.
When no expression is explicitly specified, $\bm{sizeof...(T)}$ is the implied \textit{equivalent} and the \textit{size} annotator may be omitted if not followed by the \textit{tuple} annotator.
In \textit{declarative context}, the pack identifier is specified within \textit{triple-dot} and the annotator; in \textit{expansive context}, the pack identifier precedes the \textit{triple-dot} in full compliance with current parameter pack semantics.
Whenever used in \textit{expansive context}, any expressions related to \textit{size} annotation must express pack expansion to a size within the interval specified in declaration.

\item\p The \textcolor{Magenta}{\textbf{\textit{range}}} annotator is strictly used in \textit{expansive context}, as either one or two comma separated and curly brace enclosed integral constant expressions immediately following an already specified pack identifier, without an accompanying \textit{triple-dot} specifier.
With one expression, it expands to a single parameter contained in the original pack whose \textit{index} is the enclosed expression (i.e. $\bm{T\{N\}}$).
When two expressions are used, they serve as endpoints of a left-closed, right-open interval of \textit{indices} to which the pack expands (i.e. $\bm{T\{N,M\} \equiv T\{N\},T\{N+1\},T\{N+2\},...T\{M-1\}}$).

\item\p The \textcolor{Magenta}{\textbf{\textit{tuple}}} annotator is an omissible (\textit{strictly following curly-braced annotation}) square bracket enclosed integral constant expression whose implied value is $\bm{1}$.
In \textit{declarative context} it has reflective properties upon the types comprising the pack, since it requires the pack to be an exact multiple of its first $\bm{sizeof...(T)/k}$ parameters  with \textbf{\textit{k}} being the enclosed integral constant expression.
In \textit{expansive context}, it repeats the annotated pack as many times as said expression.

\end{enumerate}
\p Annotation by \textit{size} with deducible endpoint expressions can allow specification of template parameter lists where \textit{multiple} size-annotated parameter packs coexist in declarative context with even non-annotated packs.
\textcolor{RoyalBlue}{When multiple size-annotated packs are present, futher work is needed to specify when deducibility is ensured (on the to-do list of the current branch) otherwise ambiguity ensues.}

\begin{minted}{c++}
/* 1: there is a way to allow this but complexity must be justified and properly
 *    worded to make a point for or against it */
template<typename... X{5,10}, typename... Y{3,5}> struct sample1 {}; /* see 1 */
/* 2: these are more than ok. */
template<typename... X{5,10}, typename... Y>      struct sample2 {}; /* ok */
template<typename... X{10}, typename... Y>        struct sample3 {}; /* ok */
\end{minted}


\p Annotators thus allow index and interval-based parameter pack semantics to be used in both declaration (for matching purposes) and expansion (for matching and generative purposes) and address directly \textbf{EWG\#30} \cite{Abrahams2012} using terse, unambiguous and coherent syntax that graciously degenerates into parameter packs as we know them today.
This is by design so that they can be easily submitted to \textit{partial ordering}, a well known matching criterion in C++ between them and regular parameter packs.
Any annotation where an invalid interval is specified, removes the template containing such pack from the matching candidates status, which is the SFINAE \cite{sfinae} effect in this case.

\subsection{Ternary Annotation and Generative Constructs}
    %%% 
%%% Annotated C++ template parameter packs
%%% 
%%% Copyright (C) 2014, George Makrydakis irrequietus@gmail.com>
%%%
%%% This document is released under a Creative Commons 4.0 license, specifically
%%% Attribution-NonCommercial-NoDerivatives 4.0 International. Details for the
%%% aforementioned license at http://creativecommons.org/licenses/by-nc-nd/4.0/
%%% This repository's "license.txt" contains the full text of said license.
%%%
%%% This document draft discusses the possibility of extending C++ template
%%% parameter packs with optional syntax described as "annotation". The only
%%% git repository for tracking real-time changes by the original author is
%%% at https://github.com/irrequietus/atpp.
%%%
%%% Periodically, the author ships major/important revisions of the draft in
%%% PDF format, using the project website at http://atpp.irrequietus.eu with
%%% each PDF file having a specific sha1sum checksum.
%%%
%%% This section is part of said document.

{\color{magenta}{\textit{Currently developing exposition.
Formal thinking comes first, but a series of quick examples will be properly adjusted for immediate presentation in earlier sections.
This should get a more descriptive name.}}}

        
