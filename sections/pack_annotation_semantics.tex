%%% 
%%% Annotated C++ template parameter packs
%%% 
%%% Copyright (C) 2014, George Makrydakis irrequietus@gmail.com>
%%%
%%% This document is released under a Creative Commons 4.0 license, specifically
%%% Attribution-NonCommercial-NoDerivatives 4.0 International. Details for the
%%% aforementioned license at http://creativecommons.org/licenses/by-nc-nd/4.0/
%%% This repository's "license.txt" contains the full text of said license.
%%%
%%% This document draft discusses the possibility of extending C++ template
%%% parameter packs with optional syntax described as "annotation". The only
%%% git repository for tracking real-time changes by the original author is
%%% at https://github.com/irrequietus/atpp.
%%%
%%% Periodically, the author ships major/important revisions of the draft in
%%% PDF format, using the project website at http://atpp.irrequietus.eu with
%%% each PDF file having a specific sha1sum checksum.
%%%
%%% This section is part of said document.

\textcolor{RoyalBlue}{\textit{NB: This is already in the process of getting removed with a shorter and earlier segment, but do not remove until that is finalized.}}

\begin{enumerate}
\item\p An annotated template parameter pack is an n-ordered sequence of zero or more template parameters of the same type (non-type, type-type, template-type) whose identifier is followed by the \textit{interval} and \textit{pattern} annotators.
The \textit{interval annotator} must always precede the \textit{pattern annotator} while both follow the identifier used for the \textit{annotated template parameter pack}.
The \textit{interval annotator} serves as an explicit size constraint, while the \textit{pattern annotator} serves as the explicit pattern constraint we discussed about before.

\item\p In the context of the \textit{interval annotator}, when $\bm{\{N\}}$ or $\bm{\{N,K\}}$ are used, they refer to constant expressions $\bm{N}$ and $\bm{K}$ and are referred to as \textit{interval factors} of an \textit{interval annotation}.
The first case specifies that a given pack is actually a shorthand for a parameter sequence consisting of exactly $\bm{N}$ parameters of the same kind (non-type,type-type,template-type), but not necessarily resolving to the same type during parameter to argument conversion.
When two comma-separated constant expressions are used, the \textit{interval annotator} refers to a valid right-open interval of $\bm{[N,K)\in\mathbb{Z}_{\geq 1}}$ within which the size of a pack matching the expression should be in order for the match to be valid.
When no constant expressions are used, $\bm{\overline{T}\{\}}$ is equivalent to $\bm{\overline{T}}$, $\bm{\{|\overline{T}|\}}$ and $\bm{\{0,|\overline{T}|\}}$ which abstract a parameter pack of zero or more parameters of the same kind (non-type,type-type,template-type), but not necessarily resolving to the same type during parameter to argument conversion.
This is how \textit{interval annotation} resolves into parameter pack semantics as we know them in C++11 / C++14.

\item\p When the \textit{interval} annotator is used with a single constant expression, it can be referred to as \textit{anchoring annotator} and the annotation itself referred to as \textit{anchoring annotation} regardless of whether a \textit{pattern annotator} follows or not.
\textit{Anchoring annotation} is equivalent to having declared a template parameter list with a fixed number of parameters.

\item\p In the context of the \textit{pattern annotator} $\bm{[]}$ or $\bm{[1]}$ or $\bm{[N]}$ are used for declaring an annotated parameter pack made out of either a single ($\bm{[]}$ or $\bm{[1]}$) or more ($\bm{[N]}$) repetitions of the same parameter pattern.
Thus, in \textit{pattern annotation}, the size of the pack does not get altered by what was specified in its preceding \textit{interval annotation}.
In order for such a pack to be considered in a resolution set as a viable candidate, its size during instantiation must be an exact multiple of the \textit{pattern annotation factor}, given that it must be comprised of an exact multiple of repetitions of a parameter sequence whose size \textbf{\color{red}{\textit{is}}} the \textit{pattern annotation factor}.

\item\p Annotated parameter packs co-declared with regular parameter packs in the same template parameter list gain partial ordering precedence over the latter unless they constitute equivalent forms of regular packs themselves, in which case a match cannot be made because parameter deduction cannot be guaranteed.
Multiple packs with \textit{anchoring annotation} may be present with any order in the declaration of a template parameter list.
Unlike their non-annotated and interval annotated counterparts, they are deterministically specified in a context where deducibility is guaranteed.

\item\p When \textit{anchoring annotation} is used in an expansion context it must immediately follow the unpacking ellipsis.
Using an \textit{anchoring annotator} after the pack specifier without it being expressed as expanding, results to individually accessing a parameter with an order specified by the enclosed constant expression.
The \textit{anchoring annotator} is called \textit{individual accessor} in this context.
If such an expression is invalid, the code is ill-formed.

\item\p When the \textit{pattern annotator} is used in an expansion context, it must follow the unpacking ellipsis or an eventual \textit{anchoring annotator}; it always resolves to expanding the preceding pack as many times as the \textit{pattern factor} dictates and is not indicative of any constraints upon an already declared parameter pack.

\item\p \textit{Interval annotation} can only be used in the declaration of a template parameter list.
A template instantiation or argument list involving parameter expansion of pack declared with \textit{interval annotation}, with an anchoring factor in expansion not within the interval specified by the \textit{interval annotation} used in its declaration results in its removal from the resolution set within the involved translation unit.
Invalid intervals, negative annotation factors of any kind, zero-valued anchoring and / or pattern factors have the same removal effect.
This is practically equivalent to SFINAE removal of annotated packs.

\item\p The template parameter pack specifier can be used in constant expressions of any of its annotation factors.

\end{enumerate}