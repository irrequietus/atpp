%%% 
%%% Annotated C++ template parameter packs
%%% 
%%% Copyright (C) 2014, George Makrydakis irrequietus@gmail.com>
%%%
%%% This document is released under a Creative Commons 4.0 license, specifically
%%% Attribution-NonCommercial-NoDerivatives 4.0 International. Details for the
%%% aforementioned license at http://creativecommons.org/licenses/by-nc-nd/4.0/
%%% This repository's "license.txt" contains the full text of said license.
%%%
%%% This document draft discusses the possibility of extending C++ template
%%% parameter packs with optional syntax described as "annotation". The only
%%% git repository for tracking real-time changes by the original author is
%%% at https://github.com/irrequietus/atpp.
%%%
%%% Periodically, the author ships major/important revisions of the draft in
%%% PDF format, using the project website at http://atpp.irrequietus.eu with
%%% each PDF file having a specific sha1sum checksum.
%%%
%%% This section is part of said document.

\p Using function templates and lambdas, a peculiar kind of "switch" may be implemented in {\color{red}{\textbf\textit{valid}}} C++11 / C++14 as in the following code snippet.
Specifically, lambdas and SFINAE are combined to provide a "block switch" effect within the body of the \textit{func\_templ} function template depending on the number of parameters present in the pack during instantiation.
The \textit{no-op} calls are optimized away.
\begin{minted}{c++}
template<std::size_t N, typename... X, typename F>
typename std::enable_if<(sizeof...(X) == N),void>::type sel(F&& f) { f(); }

template<std::size_t N, typename... X>
void sel(...) {} /* no-op */

template<typename... T>
typename std::enable_if<(sizeof...(T) >= 1) && (sizeof...(T) < 6),void>::type
function(T... t) {
    sel<1,T...>([&t...](){ printf("one\n"); });
    sel<2,T...>([&t...](){ printf("two\n"); });
    sel<3,T...>([&t...](){ printf("three\n"); });
    sel<4,T...>([&t...](){ printf("four\n"); });
    sel<5,T...>([&t...](){ printf("five\n"); });
}
\end{minted}

\p We can implement the previous example more easily with size annotation.
An invalid anchored pack expansion for a pack identifier declared with size annotation removes the template from the resolution candidates.
Thus, it becomes easier to use such idioms even without the \textit{std::enable\_if} C++11/C++14 template when defining the \textit{func\_tmpl} function template and its assistive functions.

\begin{minted}{c++}
template<typename...X, typename F>
void sel(F&& f) { f(); }

void sel(...) {} /* no-op */

template<typename... T{1,6}>
void func_tmpl(T... t) {
    sel<T...{1}>([&t...](){ printf("one\n"); });
    sel<T...{2}>([&t...](){ printf("two\n"); });
    sel<T...{3}>([&t...](){ printf("three\n"); });
    sel<T...{4}>([&t...](){ printf("four\n"); });
    sel<T...{5}>([&t...](){ printf("five\n"); });
}
\end{minted}


\subsection{Enhancing class template partial specializations}
\p In contrast to function templates, the ability of class templates to have partial specializations makes them more malleable to specifying patterned sequences of types upon which template meta-programming driven type pattern matching occurs.
A naive implementation for specializing a class template definition according to the size of the parameter pack declared in the template parameter list would require as many partial specializations as the sizes of interest.

\p The following is a non-naive {\color{red}{\textbf\textit{valid}}} C++11 / C++14 implementation where the partial specializations required for this kind of problem are delegated to a nested class template.
The partial specializations of the nested template are subject to removal from the resolution set through \textit{std::enable\_if}.
Unlike the approach we followed in function templates, partial specializations open the door for more complex type calculations for "block-switch" like problems.
\begin{minted}{c++}
template<typename... X>
struct classy_impl {
private:
    struct void_{};
    
    template<bool B, typename Z> /* C++14 has this as std::enable_if_t */
    using enabler = typename std::enable_if<B,Z>::type;
    
    template<typename, typename...> struct impl_{};
    
    template<typename... T>
    struct impl_<enabler<(sizeof...(T) >= 0 && sizeof...(T) < 3), void_>, T...>
    { static void deploy(){ printf("[0,3)\n"); } };
    
    template<typename... T>
    struct impl_<enabler<(sizeof...(T) >= 3 && sizeof...(T) < 6), void_>, T...>
    { static void deploy(){ printf("[3,6)\n"); } };
    
    template<typename... T>
    struct impl_<enabler<(sizeof...(T) >= 6), void_>, T...>
    { static void deploy(){ printf(">=6\n"); } };
public:
    typedef impl_<void_,X...> type;
};

template<typename... X>
using classy = typename classy_impl<X...>::type;
\end{minted}
\p If we try to solve the same problem using \textit{size annotation}, there is not even need for a nested template to exist in order to handle the necessary partial specializations.
The inherent SFINAE character of annotation semantics allows for a very succint formulation of programmer intent directly into C++ code.
Additionally, there is no need to provide for a long series of specializations even in naive implementations, making constructs depending on pattern matching through long sets of partial specializations easier to deal with.
\begin{minted}{c++}
template<typename...>
struct classy {
    static void deploy()
    { printf("sizeof...(X) >=6\n"); }
};

template<typename... X{0,3}>
struct classy<X...> {
    static void deploy()
    { printf("sizeof...(X) in [0,3)\n"); }
};

template<typename... X{3,6}>
struct classy<X...> {
    static void deploy()
    { printf("sizeof...(X) in [3,6)\n"); }
};
\end{minted}