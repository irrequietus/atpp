%%% 
%%% Annotated C++ template parameter packs
%%% 
%%% Copyright (C) 2014, George Makrydakis irrequietus@gmail.com>
%%%
%%% This document is released under a Creative Commons 4.0 license, specifically
%%% Attribution-NonCommercial-NoDerivatives 4.0 International. Details for the
%%% aforementioned license at http://creativecommons.org/licenses/by-nc-nd/4.0/
%%% This repository's "license.txt" contains the full text of said license.
%%%
%%% This document draft discusses the possibility of extending C++ template
%%% parameter packs with optional syntax described as "annotation". The only
%%% git repository for tracking real-time changes by the original author is
%%% at https://github.com/irrequietus/atpp.
%%%
%%% Periodically, the author ships major/important revisions of the draft in
%%% PDF format, using the project website at http://atpp.irrequietus.eu with
%%% each PDF file having a specific sha1sum checksum.
%%%
%%% This section is part of said document.

\p In the following example, template parameter lists containing packs are declared and expanded in class template partial specializations and function template overloads in a variety of ways fully compatible with the C++11/C++14 standard, for pattern matching purposes.
Once compiled, the resulting binary prints a series of strings on the terminal that are dependent upon the C++ templates and other features are used as computational devices during instantiation phase.

\begin{minted}{c++}
#include <cstdio>

template<typename... Types_T>
struct packed; /* class template declaration */

template<typename Head, typename... Tail>
struct packed<Head,Tail...> { /* class template partial specialization */
    typedef packed<Tail...> pop_front;
};

template<>
struct packed<> /* class template explicit specialization */
{ typedef packed<> pop_front; };

template<typename... T>
void function(packed<T...>) { /* general case */
    printf( "%lu types contained in the pack, pack has still %lu\n"
          , sizeof...(T)
          , sizeof...(T) );
    function(typename packed<T...>::pop_front());
}

template<typename T,typename...X>
void function(packed<T,T,X...>) { /* notice the repetition */
    printf( "2 identical types found, pack has still %lu \n"
          , sizeof...(X) + 2 );
    function(typename packed<T,T,X...>::pop_front());
}

template<typename X, typename Y, typename...T>
void function(packed<X,Y,X,Y,T...>) { /* notice a repeating pattern */
    printf("4 types are a pattern of 2, pack has still %lu\n", 4 + sizeof...(T));
    function(typename packed<X,Y,T...>::pop_front());
}

template<typename X, typename Y, typename...T>
void function(packed<X,Y,T...>) { /* notice the difference with T,T,X... */
    printf("2 different types found, pack has still %lu\n", 2 + sizeof...(T));
    function(typename packed<X,Y,T...>::pop_front());
}

void function(packed<>) { /* better match than packed<T...>, with T... empty */
    printf("no types contained anymore!\n");
}

int main() {
   function(packed< char, short, short, long , int
                  , char, int  , char , short, short
                  , int , int>());
   return {};
}
\end{minted}

\p The previous example shows how several behaviors of C++ class and function template features are combined for the "best match" to prevail according to which parameter pattern is more specialized in size and repetition for the template instantiation we are dealing with.

\p Currently, pack semantics do not allow us to look further into their structure unless more code is written for such an occasion.
They can be expanded only when present in a context where all of the parameters in the template parameter list they appear can be deduced and thus why they must appear last in a declarative template parameter list.
This context dependency of parameter packs is due to their omnicomprehensive character of abstracting zero or more parameters without any size or pattern constraints.