\p Provided T\{N\} is a type suitable for arrays, one wishes to have an X sized array of values of type T\{N\}.
Usually, an identifier for this is placed between the type declarator and the array notation [], however, it is at times possible to omit the identifier for brevity in our constructs so we must make take care of such an eventual collision.
One solution would be to resort to decltype for when no identifier for such an resulting array type is used as in the example below.
Asides poor readability, using decltype in such a manner makes assumptions about the constructibility of type T{N} we would not like to make.
Reminded of the use of () with reference/pointer specifiers in array types and the fact that annotators are in essence pack-identifier bound specifiers, we can use T (\{N\}) for the shorthand form.
All other ways of specifying array types using T\{N\} do not require this and it is introduced to avoid conflict with the \textit{tuple} annotator.

\begin{minted}{c++}

template<std::size_t N, typename... T{}[sizeof...(T) > N]>
void fun1(decltype(T{N}())[100]) // ugly-dangerous but legal in context
{}

template<std::size_t N, typename... T{}[sizeof...(T) > N]>
void fun2(T ({N})[100]) // when identifier ommited.
{}

template<std::size_t N, typename... T{}[sizeof...(T) > N]>
void fun3(T ({N}*)[100]) // when identifier ommited for pointers!
{}

template<std::size_t N, typename... T{}[sizeof...(T) > N]>
void fun4(T ({N}&)[100]) // when identifier ommited for references!
{}

template<std::size_t N, typename... T{}[sizeof...(T) > N]>
void fun5(T ({N}&&)[100]) // when identifier ommited for rval references!
{}

template<std::size_t N, typename... T{}[sizeof...(T) > N]>
void fun6(T{N} const (&array)[100]) // business as usual
{}

template<std::size_t N, typename... T{}[sizeof...(T) > N]>
void fun7(T{N} const (*array)[100]) // business as usual
{}
\end{minted}
