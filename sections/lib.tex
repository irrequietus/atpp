\p All of the previous examples do not go deep enough into what integral constant expressions can really be used for.
Such a feature would shine in template metaprogramming libraries beyond simple access by index.

\p The following is an implementation of a structure preserving transformation (\textit{fmap}) as a nested template alias taking advantage of private nested class templates and their partial specializations for use in template meta-programming constructs.
It can be made quite shorter; however, the purpose here is to illustrate the way through which these features combine.
In the context of these examples, \textit{atpp\_error} is a type used to specify errors, so that instantiation is always guaranteed; this is a way to design such a library and exploiting SFINAE even out of the context of this proposal.

\begin{minted}{c++}
template<std::size_t... Atpp_N{3}> // for N, M, K
struct atpp_ranged_fmap {
private:
    /* fmap_: by default this is an "error" */
    template<std::size_t...{2}, typename...>
    struct fmap_ {
        template<template<typename...> class...>
        using impl = atpp_error;
    };
    
    /* fmap_: exploiting substitutive ordering and SFINAE */
    template< std::size_t... X{2}[X{1} > X{0}]
            , typename...T{}[sizeof...(T) >= X{1}] >
    struct fmap_<X...,T...> {
        template<template<typename...> class ...Templ_T{2}>
        using impl
            = Templ_T{0}<Templ_T{1}<T{X...}>...>;
    };
    
    /* fmap_impl: non-annotated T gets lower matching priority */
    template< template<typename...> class ...Templ_T{2}
            , std::size_t...{2}
            , typename... T >
    struct fmap_impl
    { typedef atpp_error type; };
    
    /* fmap_impl: exploiting substitutive ordering and SFINAE for T */
    template< template<typename...> class ...Templ_T{2}
            , std::size_t... X{2}
            , typename... T{Atpp_N{0,2}...}[Atpp_N{2}] >
    struct fmap_impl<Templ_T...,T...>
    { typedef typename fmap_<X...,T...>::template impl<Templ_T...> type; };

public:
    template< template<typename...> class ...Templ_T{2}
            , std::size_t... X{2}
            , typename... T >
    using fmap
        = typename fmap_impl<Templ_T...,X...,T...>::type;
    
};
\end{minted}

\newpage

\p Implementing catamorphisms (left and right folds) can be slightly more complex but not bothersome.
In the example below we use the same techniques as with \textit{fmap} to implement folding.

\begin{minted}{c++}
template<std::size_t... Atpp_N{3}> // for N, M, K
struct atpp_ranged_folds {
private:
    
    template<template<typename...> class F, typename... X{2}>
    using left
        = typename F<X{0},X{1}>::type;
    
    template<template<typename...> class F, typename... X{2}>
    using rght
        = typename F<X{1},X{0}>::type;
    
    template< template<template<typename...> class, typename...{2}> class Apl_T
            , template<typename...> class F>
    struct fold_ {
        
        template<typename... Z>
        struct impl
        { typedef atpp_error type; };
        
        template<typename... Z{2}> /* exactly 2 parameters */
        struct impl<Z...>
        { typedef Apl_T<F,Z...> type; };
        
        template<typename... Z{2, sizeof...(T)}> /* > 2 parameters */
        struct impl<Z...>
             : impl<Apl_T<F,Z{0,2}...>, Z{2,sizeof...(Z)}...>
        {};
    };
    
    template< template<template<typename...> class, typename...{2}> class Apl_T
            , template<typename...> class F
            , std::size_t... X >
    struct fold_impl
    { typedef atpp_error type; };
    
    template< template<template<typename...> class, typename...{2}> class Apl_T
            , template<typename... I{}[sizeof...(I) > 1]> class F
            , std::size_t... X{2}
            , typename... T{Atpp_N{0}, Atpp_N{1}}[Atpp_N{2}] >
    struct fold_impl<Apl_T,F,X...,T...>
         : fold_<F>::template impl<T{X...}...>
    {};
    
public:
    template<template<typename...> class F, std::size_t... X{2}, typename... T>
    using foldl
        = typename fold_impl<left,F,X...,T...>::type;
    
    template<template<typename...> class F, std::size_t... X{2}, typename... T>
    using foldr
        = typename fold_impl<rght,F,X...,T...>::type;
    
};
\end{minted}

\newpage
