%%% 
%%% Annotated C++ template parameter packs
%%% 
%%% Copyright (C) 2014, George Makrydakis irrequietus@gmail.com>
%%%
%%% This document is released under a Creative Commons 4.0 license, specifically
%%% Attribution-NonCommercial-NoDerivatives 4.0 International. Details for the
%%% aforementioned license at http://creativecommons.org/licenses/by-nc-nd/4.0/
%%% This repository's "license.txt" contains the full text of said license.
%%%
%%% This document draft discusses the possibility of extending C++ template
%%% parameter packs with optional syntax described as "annotation". The only
%%% git repository for tracking real-time changes by the original author is
%%% at https://github.com/irrequietus/atpp.
%%%
%%% Periodically, the author ships major/important revisions of the draft in
%%% PDF format, using the project website at http://atpp.irrequietus.eu with
%%% each PDF file having a specific sha1sum checksum.
%%%
%%% This section is part of said document.

\p Parametric polymorphism in C++ is implemented through class and function templates whose parameter types refer to three different declarative parameter abstractions: template non-type, template type-type and template template-type parameters.
Since C++11\cite{cpp11}, a fourth kind of template parameter abstraction has been introduced, the \textit{template parameter pack}.
Such an abstraction is used for expressing an ordered sequence of zero or more parameters of one of those three parameter types.

\p In their current specification, template parameter packs do not have any declarative features that are explicitly descriptive of their size and/or actual expansion intent.
Resourceful combinations of parameter packs with constant expressions in class template partial specializations, function templates, SFINAE\cite{sfinae}, ADL\cite{Koenig96} and partial ordering are extensively deployed in several C++ template meta-programming techniques for code generation purposes \cite{Abrahams2004}.

\p This proposal explores the effects of \textit{optionally annotating C++ template parameter packs} in a backwards compatible manner with information about size, repetition and patterned expansion during their use in template parameter lists.
Such annotation can be detrimental in reducing the cost of using templates as computational constructs for code generation within a given translation unit.
Code generation techniques using template meta-programming at any level would require less source code boilerplate, as well as having reduced dependency on type-unsafe C++ preprocessor meta-programming for repetitive constructs - where current template parameter list semantics offer no other viable alternative.