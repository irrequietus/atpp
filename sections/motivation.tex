%%% 
%%% Annotated C++ template parameter packs
%%% 
%%% Copyright (C) 2014, George Makrydakis irrequietus@gmail.com>
%%%
%%% This document is released under a Creative Commons 4.0 license, specifically
%%% Attribution-NonCommercial-NoDerivatives 4.0 International. Details for the
%%% aforementioned license at http://creativecommons.org/licenses/by-nc-nd/4.0/
%%% This repository's "license.txt" contains the full text of said license.
%%%
%%% This document draft discusses the possibility of extending C++ template
%%% parameter packs with optional syntax described as "annotation". The only
%%% git repository for tracking real-time changes by the original author is
%%% at https://github.com/irrequietus/atpp.
%%%
%%% Periodically, the author ships major/important revisions of the draft in
%%% PDF format, using the project website at http://atpp.irrequietus.eu with
%%% each PDF file having a specific sha1sum checksum.
%%%
%%% This section is part of said document.

\p The explicit goal of \textit{template parameter pack annotation} is to offer an optional descriptive "field" to the current parameter pack declaration and expansion semantics, through which constraints on parameter sequences and patterns contained within a given parameter pack may be specified.
Such a feature can offer deterministic control over the actual contents of a parameter pack within the template parameter list declaration through constant expressions, \textit{without requiring breaking changes to be introduced with past and ongoing C++ standards}.

\p \textit{Template parameter pack annotation factors} are constant expressions enclosed within \textit{template parameter pack annotators} like $\mathlarger{\bm{\{\}}}$ and $\mathlarger{\bm{[]}}$ immediately following parameter pack declaration and under certain conditions in expansion.
They aim for specifying whether the presence of a pack in a template parameter list represents a valid match for instantiation to occur when the size of the pack is specific or within a left-closed, right-open interval of values (i.e. $\mathlarger{\bm{\{\}}}$ \textit{annotator}), whether it is comprised of a repeated pattern of parameters or not (i.e. $\mathlarger{\bm{[]}}$ \textit{annotator}).
Only constant expressions can be used within \textit{annotators}, including expressions using pack identifiers that have already been declared in the template parameter list of the template involved.
These semantics arise from the typical uses of class and function templates where parameter packs are involved for \textit{pattern matching purposes}, examples of which are following.
