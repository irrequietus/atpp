%%% 
%%% Annotated C++ template parameter packs
%%% 
%%% Copyright (C) 2014, George Makrydakis irrequietus@gmail.com>
%%%
%%% This document is released under a Creative Commons 4.0 license, specifically
%%% Attribution-NonCommercial-NoDerivatives 4.0 International. Details for the
%%% aforementioned license at http://creativecommons.org/licenses/by-nc-nd/4.0/
%%% This repository's "license.txt" contains the full text of said license.
%%%
%%% This document draft discusses the possibility of extending C++ template
%%% parameter packs with optional syntax described as "annotation". The only
%%% git repository for tracking real-time changes by the original author is
%%% at https://github.com/irrequietus/atpp.
%%%
%%% Periodically, the author ships major/important revisions of the draft in
%%% PDF format, using the project website at http://atpp.irrequietus.eu with
%%% each PDF file having a specific sha1sum checksum.
%%%
%%% This section is part of said document.

\p Annotated and non-annotated packs, can be combined for the explicit purpose of declarating constructs useful for manipulating typelists \cite{Czarnecki2000,Alexandrescu2001,Abrahams2004}.
Class template partial specializations deploying substitutive ordering and type accessors can be mixed with a series of constant expressions in the annotated packs used that yield interesting results.

\p The following examples illustrate one possible implementation of the typelist concept using annotated packs, with the typevector class being named such due to type accessors yielding $O(1)$ access during compilation.

\begin{enumerate}
\item\p Initially, the \textit{typevector} class is defined, along with an {\textit{error\_type}} for convenience. Definitions for three different operations, namely {\textit{at\_pos}}, {\textit{alter\_at}} and {\textit{split\_at}} are made for individual type access at a given position, changing the type at a given position and splitting the typevector into two different ones at a given position. The default result is always the {\textit{error\_type}}.Take notice that partial specializations of the \textit{at\_pos}, \textit{alter\_at}, \textit{split\_at} templates having the annotation equivalent of a regular parameter pack, would cause ambiguity to ensue because of the equivalence between $...T\{0,sizeof...(T)\}$ and $...T$.

\item\p We begin with the bare fundamentals of \textit{error\_type} and \textit{typevector}.
\begin{minted}{c++}
struct error_type {};
/* the typevector is actually just a holder for a pack */
template<typename... X> struct typevector {};
\end{minted}
\item\p The size annotator is used for declaring an annotated parameter pack in the following partial specializations of {\textit{at\_pos}}, where the constant expression of the right bound will resolve to $T\{0,0\}$ when access beyond the bounds is attempted.
This is not a valid interval, therefore this specialization will be removed from the viable candidates set due to SFINAE, with the unspecialized class template definition providing a better match. 
\begin{minted}{c++}
template<std::size_t N, typename>
struct at_pos { typedef error_type type; };

template< std::size_t N
        , typename... T{0,(N < sizeof...(T) ? sizeof...(T) : 0)}>
struct at_pos<N,typevector<T...>>
{ typedef T{N} type; };
\end{minted}

\item\p The combination of annotated and non-annotated packs can also be used for quickly altering the type parameter present at a given position in the {\textit{typevector}}.
Again, size annotation is deployed for placing invalid access through a constant expression evaluating to 0, forcing $T\{0,0\}$ when that is attempted.
\begin{minted}{c++}
template<std::size_t N, typename>
struct alter_at { typedef error_type type; };

template< std::size_t N
        , typename X
        , typename Z
        , typename... T{0,(N < sizeof...(T) ? N : 0)} /* T{0,0} ! */
        , typename... R >
struct alter_at<N,X,typevector<T...{N-1},Z,R...>>
{ typedef typevector<T...{N-1},X,R...> type; };
\end{minted}
\item\p Splitting a {\textit{typevector}} into two halves at a given position is reduced to a problem of specifying the interval bounds correctly, with two constant expressions providing the same kind of $T\{0,0\}$ mediated SFINAE safety.
\begin{minted}{c++}
template<std::size_t N, typename>
struct split_at {
    typedef error_type first_half;
    typedef error_type secnd_half;
};

template< std::size_t N
        , typename... L{0,(N < sizeof...(L) ? N : 0)}
        , typename... R >
struct split_at<N,typevector<L...{N},R...>> { /* anchored ! */
    typedef typevector<L...> first_half; /* already anchored */
    typedef typevector<R...> secnd_half; /* already anchored */
};
\end{minted}

\item\p Given that in the context of the proposal $\bm{\overline{T}\{0,|\overline{T}|\}[1] \equiv \overline{T}\{0,|\overline{T}|\}}$ while $\bm{\overline{T}\{0,|\overline{T}|\}[0] \equiv \emptyset{}}$ with the latter invoking SFINAE since it would resolve to an empty set (making the template parameter list invalid) and $\overline{T}\{0,|\overline{T}|\}\equiv{\overline{T}\{|\overline{T}|\}}\equiv{\overline{T}\{\}}$, we can exploit the tuple annotator in declaration and the \textit{individual accessor} within template definition body for an interesting rewrite of at least one of the partial specializations defined before, e.g. of {\textit{at\_pos}}:
\begin{minted}{c++}
template<std::size_t N, typename... T{}[N < sizeof...(T)]>
struct at_pos<N,typevector<T...>>
{ typedef T{N} type; };
\end{minted}
\end{enumerate}