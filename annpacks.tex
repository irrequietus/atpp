%%
%% George Makrydakis irrequietus@gmail.com>
%% Copyright (C) 2014, All Rights Reserved
%%
%% This is my first C++ proposal

\documentclass[ notitlepage
              , a4paper
              , twoside ]{article}

\usepackage{fontspec}
\usepackage[T1]{fontenc}
\usepackage{charter}
\usepackage[expert]{mathdesign}
\usepackage{amsmath}
\usepackage{amssymb}
\usepackage{mathtools}
\usepackage{polyglossia}
\usepackage{xunicode}
\usepackage{xltxtra}
\usepackage{csquotes}
\usepackage{tabularx}

\usepackage[ usenames
           , dvipsnames ]{xcolor}

\usepackage[ outer=3.4cm
           , heightrounded
           , marginparwidth=6cm
           , marginparsep=0.5cm ]{geometry}

\usepackage[pdfauthor={George Makrydakis},
            pdftitle={Annotating C++ template parameter packs},
            pdfsubject={},
            pdfkeywords={ C++
                        , templates
                        , meta-programming
                        , template parameters
                        , template parameter packs
                        , pattern generation },
            pdfproducer={XeLateX with hyperref},
            unicode=true,
            pdfencoding=auto,
            pdfcreator={Xelatex}]{hyperref}

\usepackage{newtxtext}
\usepackage{newtxmath}
\usepackage{upgreek}
\usepackage{bookmark}
\usepackage{bm}
\usepackage{listings}
\usepackage{color}
\usepackage{tikz}
\usepackage{minted}
\usepackage{tcolorbox}
\usepackage{etoolbox}
\usepackage{parskip}
\usepackage{fancyhdr}
\usepackage{textcomp}
\usepackage{mparhack}
\usepackage{draftwatermark}
\usepackage[nottoc,numbib]{tocbibind}
\usepackage[nodate]{datetime}
\usepackage{everypage} %%% solving page layout problems in different pages
\usepackage[backend=biber,sorting=nyt,natbib
           , doi=false]{biblatex}

%%% getting git revision information
\input{gitrev.tex}

%%% clearing month, day
\AtEveryBibitem{\clearfield{month}}
\AtEveryBibitem{\clearfield{day}}
%%% nobreakspace fix           
\DeclareTextCommandDefault{\nobreakspace}{\leavevmode\nobreak\ } 
            
%%% defining some colors I like using
\definecolor{tone0}{RGB}{190,58,238}
\definecolor{tone1}{RGB}{1,41,110}
\definecolor{tone3}{HTML}{002B36}

%%% the main language of this document is of course english.
\setmainlanguage{english}

%%% setting page style
\pagestyle{fancyplain}

%%% paragraph formatting
\setlength{\parskip}{0.2cm plus1mm minus1mm}
\setlength{\abovedisplayskip}{12pt}
\setlength{\abovedisplayshortskip}{12pt}
\setlength{\belowdisplayskip}{12pt}
\setlength{\belowdisplayshortskip}{12pt}

%%% settings for hyperlinks
\hypersetup{ unicode=true
           , colorlinks=true
           , linkcolor=Blue
           , citecolor=MidnightBlue
           , filecolor=magenta
           , urlcolor=magenta }

%%% activating everypage hooks for header geometry issues
\AddEverypageHook{%
    \ifthenelse{\value{page}=1}%
    {\renewcommand{\headheight}{94pt}}
    {\renewcommand{\headrulewidth}{0.0pt}}%
    {\renewcommand{\headrulewidth}{0.4pt}}%
}

%%% making 'minted' segments embedded into a colorbox           
\AtBeginEnvironment{minted}{\fontsize{8}{8}\selectfont}
    \BeforeBeginEnvironment{minted}{\begin{tcolorbox}}
\AfterEndEnvironment{minted}{\end{tcolorbox}}

\usemintedstyle{trac}

%%% A macro for referring to non-type, type, template template parameter types.
\newcommand{\typepack}[1]{$\bm{\uptau_{#1}$}}
\newcommand{\epack}[1]{$\overset{...}{#1}$}

%%% macros for numbering paragraphs, essentially the \p{} macro
\newcommand{\parnum}{\bfseries\arabic{parcount}}
\newcounter{parcount}
\newcommand\p{%
    \stepcounter{parcount}%
    \leavevmode\marginpar[\hfill\parnum]{\parnum}%
}

%%% cleanup all information related to fancy headers
\fancyhf{}

%%% set footer information
\fancyfoot[LO,RE]{\thepage{}\
\center{Copyright \textcopyright{2014} George Makrydakis\\
\texttt{\href{mailto:irrequietus@gmail.com}{irrequietus@gmail.com}}}}

%%% set header information. The first page header is slightly different
\fancyhead[LE,RO]{\ifthenelse{\value{page}=1}{{\begin{tabular}{r l}\
{\color{magenta}docnum} &  {\color{tone0} DEEMED IMMATURE} \\
{\color{magenta}project} & Programming Language C++, EWG \\
{\color{magenta}date} & \date{\today{} \currenttime{} EEST} \\
{\color{magenta}revision} & \texttt{\GitShortRev{}} \\
{\color{magenta}author} & George Makrydakis \\
{\color{magenta}reply-to} & \texttt{\href{mailto:irrequietus@gmail.com}{irrequietus@gmail.com}} \\
{\color{magenta}for the kind attention of} & {\color{red}Ville Voutilainen}\\
\end{tabular}}}{\slshape Annotating C++ template parameter packs\\
\bf{{\color{magenta} \GitLongRev{} - \date{\today{} \currenttime{} EEST}}}}}
    
%%% setting the width of the header ruler
\renewcommand{\headrulewidth}{0.2pt}
    
%%% watermark info for the draft stage
\SetWatermarkLightness{0.942}
\SetWatermarkScale{3}


\setlength{\headheight}{30pt}

%%% I do not want the language string present in each bibliography item.
\renewbibmacro{in:}{}%
\AtEveryBibitem{%
  \clearlist{language}%
}

\addbibresource{annpacks.bib}



\begin{document}

\bookmark[dest=TitlePage]{Annotating C++ template parameter packs}

\title{Annotating C++ template parameter packs}
\author{George Makrydakis\\
\texttt{\href{mailto:irrequietus@gmail.com}{irrequietus@gmail.com}}\\\bf{{\
\color{magenta}UNCORRECTED INCOMPLETE DRAFT
\date{\bf{\today{} \currenttime{} EEST}}}}}
\maketitle


\begin{abstract}
\p Parametric polymorphism in C++\cite{cpp11} is implemented through class and function templates whose parameter types refer to three different abstractions used for template declarations: template non-type, template type and template-template type parameters.
Template arguments correspond to these three kinds of parameters and replace every one of the three kinds of template parameters during instantiation phase.
A fourth kind of template parameter abstraction has been in the C++11 standard is referred to as template parameter packs, allowing variadic templates to be introduced and augmenting the expressivity of template meta-programming techniques.
The current document aims to study a series of possible extensions to template parameter pack notation in order to allow for the generation of complex sequences of type patterns in both pack declaration and expansion.
\end{abstract}

\section{Introduction}
\p Template parameter packs can be viewed as product types from a
theoretical point of view, with cohere semantics in both their declaration
and expansion within a given template parameter/ argument list. Their important
sequence processing features used with other characteristics of the C++ language
like partial class template specializations, tag dispatching, class and function
template partial ordering, SFINAE, ADL, result into advanced code generation
using C++ template meta-programming. This document explores the possibility
of implementing advanced pattern generation techniques through template
parameter packs, without introducing backwards incompatibilities and/or conflicts with
established C++ practices, through optional notation
accompanying template parameter pack declaration/expansion. In essence, optionally turn parameter packs 
into constrained pattern generators instead of plain unconstrained pattern containers.

\section{Motivation, scope, impact on the standard}

\p With the proposed approach of optional pack notation, we can reduce the code
boilerplate required for multiple class
template partial specializations and multiple function templates in order to
offer the most specialized implementation of algorithms for a particular sequence
of type patterns.Extending template parameter
pack notation without introducing backwards incompatibilities,
that if generic enough, may obviate the need for complex and constrained
uses of preprocessor meta-programming in a series of scenarios where
C++ template and constexpr meta-programming are inadequate due to the
nature of the patterns required to ge generated. Source code generation
partially assisted through preprocessor meta-programming using both
vertical and horizontal expansion/repetition patterns can be found in the
critically acclaimed and peer-reviewed boost C++ libraries.
\p In order to understand why such notation can be useful, we need to observe
certain behaviours of parameter packs with pattern matching as intended in
even simple cases of C++ template meta-programming.

\subsection{Parameter packs and Pattern Matching}
\p The following source code example is C++11/C++14 compliant and illustrates
some simple ways through which manipulation of template parameter pack declaration
and expansion are practiced. It becomes quite apparent that because of the
abstraction packs offer, variadic template parameter and argument lists can be
useful computational constructs during template instantiations. In the
following example, a template
parameter pack is declared, manipulated and expanded in a variety of ways
fully compatible with the C++11/C++14 standard. Once compiled, the resulting
binary prints a series of strings on the terminal that are dependent upon
the C++ features used as computational devices during instantiation phase.

\begin{minted}{c++}

#include <cstdio>

template<typename... Types_T>
struct packed;

template<typename Head, typename... Tail>
struct packed<Head,Tail...> {
    typedef packed<Tail...> pop_front;
};

template<>
struct packed<>
{ typedef packed<> pop_front; };

template<typename... T>
void function(packed<T...>) {
    printf("%d types contained\n", sizeof...(T));
    function(typename packed<T...>::pop_front());
}

template<typename... X, typename... Y>
void function(packed<X...,long,Y...>) {
    printf( "%d types contained and a long found!\n"
          , sizeof...(X) + sizeof...(Y) + 1);
    function(typename packed<X...,long,Y...>::pop_front());
}

template<typename T,typename...X>
void function(packed<T,T,X...>) {
    printf("%d IDENTICAL types found out of %d \n", 2, sizeof...(X)+3);
    function(typename packed<T,T,X...>::pop_front());
}

template<typename X, typename Y, typename...T>
void function(packed<X,Y,T...>) {
    printf("just %d DIFFERENT types contained\n", 2 + sizeof...(T));
    function(typename packed<X,Y,T...>::pop_front());
}

void function(packed<>) {
    printf("no types contained anymore!\n");
}

int main() {
   function(packed<char,short,short,long,short,int,int>());
   return {};
}

\end{minted}

\subsection{Context dependency of pack declaration and expansion}

\p From the example above, it is evident that pattern matching during
instantiation is evidently occuring even on specified
ordered sequences of type patterns in the left bound of a template parameter/argument
list. For this to happen, whenever template
parameter packs are involved, they have to be allowed expansion into the actual
ordered type sequences they abstract in order for things like partial
ordering to be respected. This occurs only when the context where
these packs appear in a template parameter list allows for the parameters
involved to be deduced. Let's begin with a very simple duo of variadic class
templates in the following code segment:

\begin{minted}{c++}
template<typename...>
struct wrapper {};

template<typename... T>
struct some_template {
    static void hello()
    { printf("hello world!\n"); }
};
\end{minted}


\p A few simple class template partial specializations
can help illustrating when the pack declaration and expansion are allowed. Notice
that in both specializations that follow, the template parameter pack is not
expanded within the same template parameter list where it was declared and
is ordered as the last argument used (right bound of the template argument
list of the class template partial specialization). This allows for the pack to be expanded
within a context where the totality of parameters turned arguments can be deduced,
therefore correct and specialized instantiation to occur:

\begin{minted}{c++}
template<typename... X>
struct some_template<wrapper<X...>,int> { /* X is alone! */
    static void hello()
    { printf("hello world from wrapper!\n"); }
};

template<typename... X>
struct some_template<wrapper<int,X...>,int> { /* rightmost X! */
    static void hello()
    { printf("the first type in the wrapper is an int...\n"); }
};
\end{minted}

\p The order of appearance in a given template parameter/argument list of a
parameter pack respect to non-pack template parameters is the single most
important constraint for allowing the deduction of all parameters involved
and thus pack expansion. In the following example, due to the pack appearing
before a non-pack parameter the compiler cannot deduce all the parameters
involved when the pack is required to expand in the template argument
list the following class template partial specialization:

\begin{minted}{c++}
template<typename... X>
struct some_template<wrapper<X...,int>,int> { /* cannot deduce parameters! */
    static void hello()
    { printf("this code is wrong!"); }
};

template<typename...X>
struct some_template<X...,int> { /* cannot deduce parameters! */
    static void hello()
    { printf("this code is also wrong!\n"); }
};
\end{minted}

\p This behaviour is consistent in both class template partial specializations
and function template declarations even when multiple template parameter
pack declarations exist in a given template parameter/argument list. An example
where these are deployed follows. Notice the ordering of the packs required to expand
within each instantiation of class template I, allowing for all parameters
to be deduced:

\begin{minted}{c++}
/*  Multiple pack expansion in a context where parameters can be deduced! */
template<typename... A, typename... B, template<typename...> class I>
void function(I<A...>,I<B...>)
{ printf("multiple parameter packs in function templates are ok!\n"); }
\end{minted}

\p As useful as template parameter packs are, their
declaration and expansion semantics can only currently abstract an ordered sequence
of parameters of "unconstrained" length in both scenarios. Parameter packs
cannot expand
within the same template parameter list where they are declared because of
their unbound expansion compromising parameter deduction and partial
ordering in both class and function templates. The same rule is
valid for template argument lists. However,
unsurprisingly, the following code
works because the parameter pack is required to expand within class template
scope, where the all parameters have already been deduced during
instantiation and thus the argument list of the enclosed member function
is valid.:

\begin{minted}{c++}
template<typename... X>
struct working {
    typedef wrapper<X...,int> type;
    
    static void function(X...,int) {
        printf("surprised to see me?\n");
    }
};
\end{minted}

\p In essence, template parameter pack ordering is constrained to be right-bound
in parameter and argument lists due to its unconstrained expansion
compromising the deducibility of all the parameters involved, a feature
necessary for pattern matching to occur.
By specifying certain constraints in pack declaration
and/or expansion we would be forcing the creation of specific template parameter
and argument lists in contexts where parameter deduction is unattainable. In
the simplest case, expanding a parameter pack in the left bound of a parameter/argument
list with such constraints
would yield a finite amount of parameters ("fixed pack size") of the same type
(non-type,type, template type) for which partial ordering rules would be
safe to apply. In the more general case, also decide which of the types
contained already present in the pack should expand when certain constraints are met,
allowing again the selective nature of partial ordering to occur.
\section{Design Decisions, from theory to practice}
\subsection{Symbolic notation for theoretical considerations}

\p For the purposes of this
document, template parameter packs in use within a given translation unit
will constitute a theoretical set (\typepack{\pi}) of template non-type
(\typepack{0}), template type (\typepack{1}) and template template type
parameters (\typepack{2}).  Capital letters with an overline
(e.g. $\bm{\overline{T}}$) refer to the declaration of a specific template parameter
pack. The number
of template parameters in a given $\bm{\overline{T}}$ is specified as
$\bm{|\overline{T}|}$ and will be referred to as template parameter pack magnitude.
Each $\bm{|\overline{T}|}$ abstraction can be optionally followed by a single curly brace
enclosure ($\bm{\{\}}$) with up to two comma separated symbolic identifiers which reflect
eagerly evaluated constant expressions. If and only if such additional notation
to $\bm{|\overline{T}|}$ exists, it itself may be followed by a single square
bracket enclosure ($\bm{[]}$) containing up to a single symbolic identifier  for an eagerly
evaluated constant expression. This last operator, when used with an evaluated constant
expression will refer to the times the entire pack is to be repeated.
For a given pack declaration $\bm{|\overline{T}|}$
the pack expansion operation is symbolized as $\overset{\bm.\bm.\bm.}{\bm{T}}$.
The term pack declarator is reserved for $\bm{\overline{T}\{\}[]}$, pack
generator for $\overset{\bm.\bm.\bm.}{\bm{T}}\bm{\{\}[]}$. The series
of equivalences of such notation with the current way of declaring and
expanding parameter packs are illustrated accordingly.

\begin{align}
\overline{T}\{\}[] &\equiv \overline{T}\{\} \equiv \overline{T} \equiv ...T\\
\overset{...}{T}\{\}[] &\equiv \overset{...}{T}\{\} \equiv\overset{...}{T} \equiv T...\\
|\overline{T}| &\equiv sizeof...(T)\\
\overline{T}\{|\overline{T}|\}[1] &\equiv \overline{T}\{|\overline{T}|\}[] \equiv \overline{T}\{\}[] \equiv ...T\\
\overset{...}{T}\{|\overline{T}|\}[1] &\equiv \overset{...}{T}\{|\overline{T}|\}[] \equiv \overset{...}{T}\{\}[] \equiv T...
\end{align}

\p One way to represent the n-tuple character of template parameter
packs is to model them as nested ordered pairs. The following representation is
congruent with the way template parameter packs are being manipulated using
pattern matching techniques, from their left to their right bound (head to tail).
The whole concept is of course reminiscent of a right fold of a cons function
over our ordered n-sequence of types.

\begin{align}
\overline{T}\{|\overline{T}|\}
    &= \underbrace{( t_{0}
                   , ( t_{1}
                     , (t_{2}
                     , ( \dots{}
                      , ( t_{n-1},\emptyset)\dots))))}_{n=|\overline{T}|}
\end{align}

\p The meaning of the curly braces and square brackets can be generalized based on
our previous assumptions and illustrated using the same symbolic notation. Notice
that in this case, we end up making full use of the new pack operators, without
changing at all the semantics of the original pack declaration and expansion
as they are in the current C++ standard. The meaning of the new pack operators
is the generalization of the pack concept.
\begin{align}
\overline{T}\{n\}[1]
    &= \underbrace{( t_{0}
                   , ( t_{1}
                     , (t_{2}
                     , ( \dots{}
                      , ( t_{n-1},\emptyset)\dots))))}_{|\overline{T}|=n\times{1}} \\
\overline{T}\{n\}[m]
    &= \underbrace{( t_{0}
                   , ( t_{1}
                     , (t_{2}
                     , ( \dots{}
                      , ( t_{n-1},\overline{T}\{n\}[m-1])\dots))))}_{|\overline{T}|=n\times{m}} \\
\overline{T}\{i,j\}[1]
    &= \underbrace{( t_{i}
                   , ( t_{i+1}
                     , (t_{i+2}
                     , ( \dots{}
                      , ( t_{j-1},\emptyset)\dots))))}_{|\overline{T}|=(j-i)\times{1}} \\
\overline{T}\{i,j\}[n]
    &= \underbrace{( t_{i}
                   , ( t_{i+1}
                     , (t_{i+2}
                     , ( \dots{}
                      , ( t_{j-1},\overline{T}\{i,j\}[n-1])\dots))))}_{|\overline{T}|=(j-i)\times{n}}
\end{align}

%%% -- This is an important paragraph
\p

\begin{align}
\underset{\forall n,i,j \in \mathbb{Z}_{\geq0} \mid{} 0 \leq i \leq j}{|\overline{T}\{i,j\}[n]|} &= (j-i)\times{n}
\end{align}


\p There can be a generalized form of declaration pack magnitude based on these observations.
Notice that we are not dealing with the magnitude
of expansion, but that of declaration for the time being. The context-dependency of
$\overline{T}\{i,j\}[n]$ and $\overset{...}{T}\{k,l\}[m]$ in template parameter/argument
list declaration and within the definition body of an eventual template may allow
us a more creative use of pack extension magnitudes. But for a pack declaration, with
the proposed operators, there is a specific relationship of the pack magnitude and
the eagerly evaluated constant expressions used with the operators.

\begin{align}
\underset{\forall n,i,j \in \mathbb{Z}_{\geq0} \mid{} 0 \leq i \leq j}{|\overline{T}\{i,j\}[n]|} &= (j-i)\times{n}
\end{align}

\section{Accessors and typelists}

\subsubsection{Accessing individual elements in annotated and non-annotated packs}
\p Following pack declaration and after matching has occured, there is the problem of specifying individual access to elements being contained in said expansion.
The semantic context of the interval annotator is overloaded to be that of an individual element accessor in the definition body of a class or function template.
Individual access is specified by the annotated pack identifier without ellipsis surrounding it, followed by anchor annotation whose constant expression is the index of the type we wish to refer to.

\p Within the definition body of a function template, class template and any combination of functions, classes, whether in template form or not that are currently allowed in C++ to use as members whenever applicable annotated expansion can be used in anchor or accessor form.
The following examples illustrate how this can work without causing conflicts of any kind, given substitutive ordering.
\begin{enumerate}
\item \p Given a variadic class template with an annotated template pack $\overline{T}\{3,6\}$, write the partial class specialization gaining matching priority for a magnitude $|\overline{T}| = 5$ containing a static function accepting the extended parameter pack in its argument list and returning std::tuple with the pack expanded in reverse order.
\begin{minted}{c++}
template<typename... T{3,6}>            /* interval of [3,6) */
struct class_template<T...{5}> {        /* match when sizeof...(T) == 5 */
    /* std::tuple for T...{5}, reversed */     /* T... can be used! */
    static std::tuple<T{4},T{3},T{2},T{1},T{0}> function(T...{5}) {
        T{0} t0; /* example of a declaration within definition */
                 /* implementation here */
    }
};
\end{minted}
\item \p Another example using a $\overline{T}\{4,6\}$ pack where by convention we intend to pass template parameter types unique to each other is the following. Notice that if what is abstracted in $\overline{T}\{4,6\}$ is not really unique, member functions with the same signature will produce ambiguous overloads, as expected by the current C++ standard.
\begin{minted}{c++}
template<typename... T{4,6}>
struct class_template<T...{4}> {
    static void member_function(T{0}) {}
    static void member_function(T{1}) {}
    static void member_function(T{2}) {}
    static void member_function(T{3}) {}
};
\end{minted}
\item\p Anchored pack expansion is NOT legal to use for any specified anchor expression if an anchor has already been set in a dependent template parameter list, because that would make the parameters involved non-deducible.
\begin{minted}{c++}
template<typename... T{4,6}, typename... X>
struct class_template<T...{5}, X...> /* anchored to 5  for ...T{4,6} */
     : class_template<T...{4}>       /* error, pack has already been anchored */
{};
\end{minted}
\item\p Individual accessors are not anchors and they are not ambiguously annotated with any accompanying ellipsis, so class template partial specialization who have already used anchored annotation in expansion can mix freely with individual accessors provided an interval anchor has already been set in an annotated expansion.
\begin{minted}{c++}
template<typename... T{4,6}, typename... X>
struct class_template<T...{4}, X...>        /* anchored to 4  for ...T{4,6} */
     : class_template<T{1},T{2},T{3},X...>  /* ok, accessor expressions */
{ /* doing something interesting with T{0} here... */ };
\end{minted}
\item\p An individual accessor and an anchored expansion for the same annotated pack can coexist in the same parameter list, but not with the interval annotation they are derived from.
This is congruent to the character of standard compliant pack declaration and expansion.
\begin{minted}{c++}
 /* This is legal, T...{5} is "fixed", accessors and anchor not in the same
  * template parameter list as the annotated interval pack declaration */
template<typename...T{2,6}>
struct class_template<T...{5}> {
    static void function(T{4},T...{5},T{4}) {}
 /* static void function(T{4},T...,T{4}) {} is equivalent; anchor = 5 already */  
};
\end{minted}
\end{enumerate}
\subsubsection{Motivating example: annotated packs in typelists}
\p Annotated and non-annotated packs, can be combined for the explicit purpose of declarating constructs useful for manipulating typelists.
Class template partial specializations deploying substitutive ordering and type accessors can be mixed with a series of constant expressions in the annotated packs used that yield interesting results.

\p The following examples illustrate one possible implementation of the typelist concept using annotated packs, with the typevector class being named such due to type accessors yielding $O(1)$ access during compilation.

\begin{enumerate}
\item\p Initially, the \textit{typevector} class is defined, along with an {\textit{error\_type}} for convenience. Definitions for three different operations, namely {\textit{at\_pos}}, {\textit{alter\_at}} and {\textit{split\_at}} are made for individual type access at a given position, changing the type at a given position and splitting the typevector into two different ones at a given position. The default result is always the {\textit{error\_type}}.Take notice that partial specializations of the \textit{at\_pos}, \textit{alter\_at}, \textit{split\_at} templates having the annotation equivalent of a regular parameter pack, would cause ambiguity to ensue because of the equivalence between $...T\{0,sizeof...(T)\}$ and $...T$.

\item\p We begin with the bare fundamentals of \textit{error\_type} and \textit{typevector}.
\begin{minted}{c++}
struct error_type {};
/* the typevector is actually just a holder for a pack */
template<typename... X> struct typevector {};
\end{minted}
\item\p The interval annotator is used for declaring an annotated parameter pack in the following partial specializations of {\textit{at\_pos}}, where the constant expression of the right bound will resolve to $T\{0,0\}$ when access beyond the bounds is attempted.
This is not a valid interval, therefore this specialization will be removed from the viable candidates set due to SFINAE, with the unspecialized class template definition providing a better match. 
\begin{minted}{c++}
template<std::size_t N, typename>
struct at_pos { typedef error_type type; };

template< std::size_t N
        , typename... T{0,(N < sizeof...(T) ? sizeof...(T) : 0)}>
struct at_pos<N,typevector<T...>>
{ typedef T{N} type; };
\end{minted}

\item\p The combination of annotated and non-annotated packs can also be used for quickly altering the type parameter present at a given position in the {\textit{typevector}}.
Again, the interval annotation is deployed for placing invalid access through a constant expression evaluating to 0, forcing $T\{0,0\}$ when that is attempted.
\begin{minted}{c++}
template<std::size_t N, typename>
struct alter_at { typedef error_type type; };

template< std::size_t N
        , typename X
        , typename Z
        , typename... T{0,(N < sizeof...(T) ? sizeof...(T) : 0)} /* T{0,0} ! */
        , typename... R >
struct alter_at<N,X,typevector<T...{N-1},Z,R...>>
{ typedef typevector<T...{N-1},X,R...> type; };
\end{minted}
\item\p Splitting a {\textit{typevector}} into two halves at a given position is reduced to a problem of specifying the interval bounds correctly, with two constant expressions providing the same kind of $T\{0,0\}$ mediated SFINAE safety.
\begin{minted}{c++}
template<std::size_t N, typename>
struct split_at {
    typedef error_type first_half;
    typedef error_type secnd_half;
};

template< std::size_t N
        , typename... L{0,(N < sizeof...(L) ? sizeof...(L) : 0)}
        , typename... R{N,(N < sizeof...(R) ? sizeof...(R) : 0)} >
struct split_at<N,typevector<L...{N-1},R...{sizeof...(R)}>> { /* anchored ! */
    typedef typevector<L...> first_half; /* already anchored */
    typedef typevector<R...> secnd_half; /* already anchored */
};
\end{minted}

\item\p Given that in the context of the proposal $\bm{\overline{T}\{0,|\overline{T}|\}[1] \equiv \overline{T}\{0,|\overline{T}|\}}$ while $\bm{\overline{T}\{0,|\overline{T}|\}[0] \equiv \emptyset{}}$ with the latter invoking SFINAE since it would resolve to an empty set (making the template parameter list invalid) and $\overline{T}\{0,|\overline{T}|\}\equiv{\overline{T}\{|\overline{T}|\}}\equiv{\overline{T}\{\}}$, we can exploit the pattern annotator for an interesting rewrite of at least one of the partial specializations defined before, e.g. of {\textit{at\_pos}}:
\begin{minted}{c++}
template<std::size_t N, typename... T{}[N < sizeof...(T)]>
struct at_pos<N,typevector<T...>>
{ typedef T{N} type; };
\end{minted}
\end{enumerate}

\subsubsection{On an eventual overload of the repeater pack annotator}

\p While for the time being we have focused on this using modified symbolic notation, there is something else to be said about the ($[]$) operator.
Given that currently a C++ template parameter pack has template parameter types of the same "kind" (non-type, type, template-type) it is consequential to assume that a class template with a template parameter in its template parameter list providing a valid match for that "kind", can be used as a template meta-function for the entirety of the pack.
One exciting possibility would be to allow the contextual overload of ($[]$) to not just be a repeater, but to accept a class templates instead of constant expressions, allowing for fundamental but high order operations like folds to occur. At the current status of this draft, the semantics of such a possibility are underway.

\subsection{Simple pack annotation examples}
\subsubsection{Annotated parameter packs and substitutive ordering}
\p Every annotated pack is declared by a preceding ellipsis (symbol: $\bm{...}$) and its identifier followed by the janus annotator (symbol: $\bm{\{\}}$).
The latter either contains a single eagerly evaluated constant expression or two such expressions separated by a comma.
It is called anchoring annotator in the first case, interval annotator in the second case.

\p Either form is optionally followed by the pattern annotator (symbol: $\bm{[]}$), containing a single eagerly evaluated constant expression equal to the times the ordered sequence of types described by the janus annotator is to be repeated in order to have a match. Both annotators have empty forms that are equivalent to $\{0,sizeof...(T)\}$ and $\{sizeof...(T)\}$ for the empty janus annotator and $[1]$ for the empty pattern annotator.

\p Templates with parameter lists containing annotatively declared packs have partial ordering priority to their variadic pack counterparts, even when their annotation is equivalent to variadic packs.
The effect of annotation is substitution, due to an annotated parameter pack expressing explicit intent on behalf of the programmer to contain a specified number of types, even if that means substituting an equivalent pack declared non-annotatively.
This is called {\textbf{\em{substitutive ordering}}}.

\p Due to this substitution, any combination of $\{0,0\}$, $\{0\}$ with $[0]$ result to SFINAE being invoked.
These mean that the parameter pack cannot be declared through the janus annotators, while it cannot be repeated even once for the zero pattern.
SFINAE is also being invoked when the interval annotator does not specify a valid left-closed, right-open interval of integers greater than or equal to zero.
A candidate match is then successful if by partial ordering any other template containing non-annotated packs in equivalent placement can be found.

\p The following code snippets give the reader an overview of how annotated pack declaration works through parameter list examples.
\begin{enumerate}
\item\p This first trio of template parameter lists are equivalent and would cause an ambiguous match if any combination of the two was found within a source file, but they would match over the non-annotated parameter packs despite them being variadic because of substitutive ordering.
\begin{minted}{c++}
/*01*/ template<typename...T{}[]>     /* matches over template<typename... T> */
/*02*/ template<typename...T{}[1]>    /* matches over template<typename... T> */
/*03*/ template<typename...T{}>       /* matches over template<typename... T> */
\end{minted}

\item\p For the same reasons, this is valid for every annotated pack. Substitutive ordering enforces the same preferential priority when packs of non-type, type, template-template type parameters are being declared annotatively over their variadic counterparts.
\begin{minted}{c++}
/*04*/ template<typename...T{5}>      /* fixed amount of 5 types */
/*05*/ template<typename...T{5}[2]>   /* fixed of 5, match if 1 repeat */
/*06*/ template<typename...T{3,9}[2]> /* [3, 9), match if 1 repeat */
/*07*/ template<typename...T{3,9}>    /* provide a match for [3,9) types */
/*08*/ template<template<typename...T{}[]>     class>
/*09*/ template<template<typename...T{}[1]>    class>
/*10*/ template<template<typename...T{}>       class>
/*11*/ template<template<typename...T{5}>      class>
/*12*/ template<template<typename...T{5}[2]>   class>
/*13*/ template<template<typename...T{3,9}[2]> class>
/*14*/ template<template<typename...T{3,9}>    class>
/*15*/ template<template<typename...T{}[]>     class...P{}[]>
/*16*/ template<template<typename...T{}[1]>    class...P{3}[2]>
/*17*/ template<template<typename...T{4}>      class...P{4}>
/*18*/ template<template<typename...T{5}>      class...P>
/*19*/ template<template<typename...T{5}[2]>   class...P{1,3}[4]>
/*20*/ template<template<typename...T{3,9}[2]> class...P{}[2]>
/*21*/ template<template<typename...T{3,9}>    class...P{2}>
\end{minted}
\end{enumerate}

\subsubsection{Substitutive ordering in declaration and expansion}
\p By extending the field where partial ordering applies, annotated packs allow for the introduction of substitutive ordering.
Such ordering means that annotated packs gain matching precedence over undetermined packs as representing the set of all deterministic type-safe expansions of an undetermined pack with whose rank equals that of the annotation interval.
There are some constraints that must be enforced in order for pack annotation to be compatible with existing C++11/C++14 practices.

\begin{enumerate}
\item Annotation semantics are dependent of whether they are used in pack declaration or expansion.
Annotated declaration constrains annotated expansion to be of magnitude kind.
When magnitude expansion is not specified in a annotatively declared pack, undetermined notation is used or the regular unpacking is required to be specified in the template argument list involved.
\item Expansive annotation can only be used in the context of the declaration of a template parameter list that is not the same as the template parameter list where the annotated pack was declared. This is in accordance with how parameter pack declaration and expansion behave.
\item Expansive annotation cannot be used within a definition body, with undetermined notation or regular unpacking being specified for the use of the entire ordered sequence of types the annotated pack is abstracting.
\item Packs that were declared annotatively and subsequently used in template definitions (i.e. outside the context of a template parameter list) can use the interval annotator with a single constant expression, the latter serving as access to individual elements contained, with value within $[0,|\overline{T}\{i,j\}[n]|)\subset{\mathbb{Z}_{\geq 0}}$.
\item Class templates can use interval, anchor and repeater annotation for parameter packs in their template parameter or argument list declarations, because they are allowed to partially specialize in C++.
\item Class templates deploying interval annotation for their parameter packs are supposed to be partially ordered with partial specializations for every anchored expansion valid within the declared interval annotation range.
If said partial specializations do not exist, matching is delegated to an appropriate, more generic match for the same declarared template identifier.
\item Function templates can use anchor and repeater but not interval annotation for parameter packs in their template parameter or argument list declarations, because they are not allowed to partially specialize in C++.
\item Expansive annotation used in a template parameter list cannot override the constraints set to a pack during annotated declaration in either class or function templates.
When annotated pack declaration uses full interval annotation, expansive annotation is allowed with the interval annotator being allowed to contain a single eagerly evaluated constant expression representing a fixed size of parameters the pack is to expand to, with the former being within the interval specified in annotated pack declaration.
\begin{minted}{c++}
template<typename...T>
struct class_template { /* implementation here */ };

template<typename...T{2,10}[3]>  /* partial specialization of class_template */
struct class_template<T...{4}> { /* 4 is in [2,10), match if repeated 3 */ };

template<typename...T{2,10}[2]>  /* partial specialization of class_template */
struct class_template<T...{4}> { /* 4 is in [2,10), match if repeated 2 */ };

template<typename...T{2,10}>  /* partial specialization of class_template */
struct class_template<T...{4}> { /* match if just 4, already in [2,10) */ };
\end{minted}
\item \p The same expansion semantics hold for function templates when full interval notation has been used in their annotated pack declaration.
This is not equivalent to partially specializing function templates, which would be illegal, because in this case the annotated expansion serves as confirming deterministically which expansion is the interval
annotation is specified for.
\begin{minted}{c++}
template<typename...T>
void function(T...) { /* match if nothing else matches */ }

template<typename... T{2,5}[2]>
void function_template(T...{3}) { /* 3 in [2,5) => match if repeats 2 */ }

template<typename... T{2,5}>
void function_template(T...{4}) { /* 4 is in [2,5) => match */ }

template<typename... T{4}>
void function_template(T...) { /* sizeof...(T) is locked to 4, match when 4 */}

template<typename... T{4}[2]>
void function_template(T...) { /* locked to 4, match when repeated 2 */}
\end{minted}
\p The following are examples where function templates with annotated parameter packs are illegal to use in the context of the proposal.
This happens when the interval annotator is used without befitting anchor notation used in expansion.
\begin{minted}{c++}
template<typename... T{2,5}>
void function_template(T...{9}) { /* 9 NOT in [2,5) => error */}

template<typename... T{2,10}>
void function_template(T...) { /* illegal, error */}
\end{minted}
\end{enumerate}

\section{Technical specification}
\subsection{Grammar Additions}
\p The current section you are reading is under revision prior to initial public release.
Not yet ready for draft stage.
Each section presented should not be affecting sections that are already presented in the current draft. Unreleased sections that are under revision could be having incomplete elaboration of the concepts involved and that is the primary reason for their revision.
\section{Acknowledgements}
The author would like to thank Ville Voutilainen, current C++ Standard Commitee EWG Chairman for strongly motivating the drafting of this document.
The "fixed pack size" initiative was originally presented on the C++ std-proposals mailing list by Maurice Bos\textsuperscript{citation pending}.

\addcontentsline{toc}{section}{References}
\printbibliography
\end{document}